\chapter*{Заключение}						% Заголовок
\addcontentsline{toc}{chapter}{Заключение}	% Добавляем его в оглавление

%% Согласно ГОСТ Р 7.0.11-2011:
%% 5.3.3 В заключении диссертации излагают итоги выполненного исследования, рекомендации, перспективы дальнейшей разработки темы.
%% 9.2.3 В заключении автореферата диссертации излагают итоги данного исследования, рекомендации и перспективы дальнейшей разработки темы.
%% Поэтому имеет смысл сделать эту часть общей и загрузить из одного файла в автореферат и в диссертацию:
В диссертационной работе выполнены экспериментальные капиллярной турбулентности на поверхности жидкого водорода и воды. Выполнено исследование генерации вихревого движение на свободной поверхности жидкости волнами движущимися под углом друг к другу.

1. Впервые наблюден переход от степенного в инерционном интервале спектра Колмогова-Захарова к “квазипланковскому” распределению $\omega^{-s}e^{-\omega/\omega_d}$ в области диссипации для капиллярной турбулентности. Экспоненциальный спад в области диссипации $\omega/\omega_d \gg 1$ соответствует теоретическому ожиданию и качественно соответствует численным вычислениям\todo{проверить ссылку!!!} \cite{Brazhnikov_IET}. Граница вязкого затухания $\omega_d$ растет с увеличением амплитуды накачки и зависит от средней высоты волны $\eta_0$ на частоте накачки как $\omega_d \sim \eta^{0.85 \pm 0.05}$. Однако наблюденная зависимость отличается от ожидаемой, показатель степени почти в три раза больше, чем предсказанное значение.

2. Экспериментально показано, что при возбуждении турбулентного состояния на поверхности воды монохроматической или широкополосной накачкой высокочастотный край инерционного интервала и характерная частота в диссипативной области отличаются в несколько раз и качественно одинаково повышаются с ростом амплитуды накачки по степенному закону с показателем степени, близким к теоретически оцененному значению для монохроматического возбуждения. В случае широкополосной накачки наблюдается значительное расхождение между экспериментальными и теоретически оцененными значениями показателя $\beta$.

3. Открыт новый механизм генерации поверхностной завихренности поверхностными волнами распространяющимися под углом друг к другу. Показано, что генерация вихревого движения на поверхности жидкости не является специфической чертой неустойчивости Фарадея.

4. Экспериментально подтверждена теоретическая модель генерации вихревого движения перпендикулярными волнами на поверхности жидкости как для капиллярных, так и для гравитационных волн. Экспериментально подтверждены амплитудная и фазовая зависимости завихренности. Амплитуда завихренности на поверхности квадратично зависит от амплитуды волн. 

5. Экспериментально наблюдена передача энергии от системы вихрей, образующих решетку, в большие вихревые масштабы.
%Экспериментально показано, что стоячие волны на поверхности жидкости в сосуде, который совершает гармонические колебания в вертикальном направлении с амплитудой переменного ускорения ниже порога параметрической неустойчивости, могут генерировать вихревое течение. В квадратном сосуде структура вихревого движения имеет вид квадратной решетки с периодом, равным длине стоячих волн. В цилиндрическом сосуде вихревое движение наблюдается только при возникновении азимутальных мод, которые возможны при амплитудах накачки выше порога рога параметрической неустойчивости. Искусственное понижение симметрии цилиндрического сосуда, которое разрешает генерацию азимутальных мод при малых амплитудах накачки, позволяет формировать вихревое движение при накачке значительно ниже порога параметрической неустойчивости Фарадея. Исходя из этих наблюдений и принимая во внимание степенную зависимость завихренности от амплитуды волн, можно утверждать, что в сосудах разной симметрии вихревое движение возникает тогда, когда на поверхности жидкости распространяется пара волн с неколлинеарными волновыми векторами.
% В частности этот механизм приводит к перемешиванию поверхности, которое может быть характерезовано как диффуззионным коэффициентом $D$ соответсующиего четвертому порядку волновой амплитуды, так из прямым действием волн на поверхности жидкости \cite{Falkovich, Buhler}. Данная задача требует дальнейшего исследования. Несмотря на то, что эксперименты производились в капиллярной области, наши теоретические построения также применимы и в на гравитационной области. Для примера соответствующая теория может быть использована для анализа вихревого движения на поверхности океана.

%Основные результаты работы заключаются в следующем.
%\input{common/concl}
%И какая-нибудь заключающая фраза.
%
%Последний параграф может включать благодарности.  В заключение автор
%выражает благодарность и большую признательность научному руководителю
%Иванову~И.И. за поддержку, помощь, обсуждение результатов и научное
%руководство. Также автор благодарит Сидорова~А.А. и Петрова~Б.Б. за
%помощь в работе с образцами, Рабиновича~В.В. за предоставленные
%образцы и обсуждение результатов, Занудятину~Г.Г. и авторов шаблона
%*Russian-Phd-LaTeX-Dissertation-Template* за помощь в оформлении
%диссертации. Автор также благодарит много разных людей и
%всех, кто сделал настоящую работу автора возможной.
