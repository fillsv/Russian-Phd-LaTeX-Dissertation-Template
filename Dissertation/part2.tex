\chapter{"Квазипланковский" спектр капиллярной турбулентности на поверхности жидкого водорода}
\section{Экспериментальная методика} %\label{sect1_1}
 Экспериментальная установка состоит из оптической ячейки расположенной в вакуумной полости гелиевого криостата и оптической системы регистрации колебаний на поверхности жидкости. Проводящий цилиндрический сосуд 6 мм глубиной и 60 мм диаметром был установлен внутри ячейки, а проводящая пластина зафиксирована в 4 мм над сосудом. Газообразный водород сконденсирован в сосуде. Радиоактивная пластина, расположенная в нижней части сосуда, ионизирует жидкий водород. В присутствии постоянного напряжения около 1 кВ между сосудом и верхней пластиной, ионы разделяются и заряды со знаком соответствующим полярности постоянного напряжения собираются под поверхностью жидкого водорода. Волны на поверхности жидкого водорода возбуждаются добавочное переменное напряжение с максимальной амплитудой 100 В.

	Использование электрического поля для создания капиллярных волн дает большое преимущество по сравнению с использованием техник возбуждения волн на поверхности воды, например неустойчивость Фарадея. Это позволяет нам возбуждать поверхность аккуратно достаточно хорошо контролируемой силой. В этой серии экспериментов в качестве переменного возбуждающего напряжения были использованы низкочастотные случайные сигналы. Эти сигналы были синтезированы через фурье-преобразование случайного набора фаз и прямоугольного амплитудного спектра, которые везде равен нулю кроме заданного частотного интервала(интервал накачки).

	Для регистрации волн на поверхности жидкости использовался метод отражения лазерного луча. Лазерный луч падает под малым скользящим лучом(около 0.2 рад) на поверхность жидкости, отражается и фокусируется линзой на фотодетектор. Напряжение на фотодетекторе усиливается и оцифровывается 24 битным аналого-цифровым преобразователем с частотой дискретизации около 100 кГц. Волны регистрировались в режиме "широкого луча", когда размер лазерного луча больше, чем характерная длина волны. Энергия отраженного лазерного луча $P(t)$  в этом режиме пропорциональна отклонению поверхности $\eta(t)$ [8]. По этой причине в дальнейшем не делается разницы между спектром корреляционной функции отклонения поверхности $<|\eta_\omega^2|>$ и энергии отраженного лазерного луча $<P_\omega^2>$ .

	Максимальная крутизна волны которая может быть зарегистрирована в эксперименте ограничена размером оптических окон криостата и приблизительно равно 0.05 рад.

\section{Экспериментальные результаты и обсуждение} %\label{sect1_1}
 Капиллярные волны возбуждались случайной силой в частотном диапазоне 39-103 Гц. Средний квадрат возбуждающего напряжения менялся от $V_p = 0$ В, т.е. отсутствие накачки, до $V_p = 30$ В , ограничение связано с максимальной крутизной волны. На рис показан пример фурье спектра для отраженной энергии лазерного луча $P_\omega^2$ для разных амплитуд возбуждающей силы. На рис хорошо видно область накачки на низкочастотной части спектра. За областью накачки следует инерционный интервал - относительно широкая частотная область, где видна степенная зависимость спектра $P_\omega^2$. Ширина инерционного интервала зависит от амплитуды накачки. Когда поверхность возбуждается слабо (переменное напряжение $V_p = 4$ В) диссипация начинается рядом с областью накачки и инерционный интервал не наблюдается. Увеличение силы накачки приводит к уширению инерционного интервала, высокочастотная граница инерционного интервала $\omega_b$ смещается к высоким частотам. Наиболее широкий инерционный интервал с границами от $\approx 0.3$ кГц, до $\omega_b \approx 4$ кГц наблюдается при максимальном напряжении накачки $V_p = 30$ В. На частотах выше высокочастотной границы колебания поверхности затухают из-за вязких потерь, кривая $P_\omega^2$ идет вниз гладко и уходит ниже уровень аппаратных шумов.

	Турбулентные спектры перестроенные в линейном масштабе на рис показывают, что убывание амплитуд волн с частотой после высокочастотной границы инерционного интервала может быть достаточно хорошо описано экспоненциальным затуханием $P_\omega^2 \sim	e^{-\omega/\omega_d}$ в некотором интервале. Подгонка согласуется с начальным предположения, что $\omega \gg \omega_d$, полученный параметр $\omega_d$ значительно меньше, чем частоты из интервала подгонки. Например спектр при напряжении накачки $V_p = 26$ В подгонялся в диапазоне 5-9 кГц с $\omega_d \approx 0.6$ кГц. К сожалению узкий интервал подгонки не позволяет установить показатель степени s “квазипланковского” распределения достаточно точно. Полученные значения $\omega_d$ в несколько раз меньше, чем видимая граница между инерционным интервалом и диссипативной области(см рис). Это несоответствие можно отнести к определенной степени свободы в определении граничной частоты, которая может быть перенормирована с помощью некоторой константы.

	Граница вязкого затухания $\omega_d$ наблюденная с помощью подгонки экспоненциального затухания в диссипативной области растет с увеличением возбуждающей силы. Для измерения уровня возбуждения использовался отклик поверхности $\eta_0$, а именно абсолютное значение $P_\omega$ на частоте 53 Гц (положение максимума распределения $P_\omega^2$ внутри области накачки). Величина $\eta_0$ прямопрорциональна средней высоте волны на той же самой частоте. На рис показано, что зависимость граничной частоты от величины возбуждения может быть описана степенным законом $\omega_d(\eta_0) \sim	\eta_0^m$ со значение показателя $m = 0.85 \pm 0.05$. Необходимо заметить, что подгонка экспоненциальных спектров с помощью “квази-Планка” с малым ненулевым $s$ ($|s| \le 2$) слабо влияет на полученный параметр $\omega_d$ (меньше чем на 20\%). Однако эта поправка не изменит показатель степени $m$ в пределах погрешности.

	Наблюдаемый показатель $m \approx 0.85$ значительно отличается от ожидаемого $m = 12/5$ из формулы (3). Стоит отметить, что в случае турбулентных каскадов, возбужденных монохроматической силой, измеренная граничная частота находится в хорошем соответствии с ожиданиями $\omega_d(\eta) \sim \eta^{1.3}$.

\section{Выводы}% \label{sect2_4}

 	Впервые наблюден переход от степенного в инерционном интервале спектра Колмогова-Захарова к “квазипланковскому” распределению $\omega^{-s}e^{-\omega/\omega_d}$ в области диссипации для капиллярной турбулентности. Экспоненциальный спад в области диссипации $\omega/\omega_d \gg 1$ соответствует теоретическому ожиданию и качественно соответствует численным вычислениям [7]. Граница вязкого затухания $\omega_d$ растет с увеличением амплитуды накачки и зависит от средней высоты волны $\eta_0$ на частоте накачки как $\omega_d \sim \eta^{0.85 \pm 0.05}$. Однако наблюденная зависимость отличается от ожидаемой, показатель степени почти в три раза больше, чем предсказанное значение.



\clearpage
