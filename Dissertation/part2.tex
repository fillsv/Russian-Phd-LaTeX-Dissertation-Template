\chapter{Турбулентность в системе капиллярных волн на поверхности воды}

Турбулентность в системе волн наряду с вихревой турбулентностью играет значительную роль во многих процессах, происходящих на Земле и во Вселенной. Она является объектом интенсивных исследований во многих системах: на поверхности океана, в атмосфере, в плазме [1]. Турбулентноcть на поверхности воды в гравитационно-капиллярном интервале частот изучалась многими исследователями в течение нескольких последних десятилетий [2–5]. Для возбуждения волн использовали различные методики. Так, в [2] применяли специальные лопатки (мешалки), погруженные в жидкость. Однако в большинстве работ для генерации волн используют параметрическую неустойчивость поверхности жидкости, совершающей вынужденные вертикальные колебания с ускорениями выше некоторого порогового значения (неустойчивость Фарадея) [3–5]. Отличительной чертой этой методики является высокий уровень возбуждения волн сразу после возникновения неустойчивости на поверхности. Такая особенность методики возбуждения не позволяет работать с волнами малой амплитуды. Кроме того, как выяснилось, при сильном возбуждении наряду с нелинейным вза- имодействием волн наблюдается генерация вихревого движения [6, 7]. Недавно в [8, 9] было показано, что завихренность формируется в результате взаимодействия нелинейных волн, имеющих непараллельные волновые векторы, т.е. в двумерном пространстве волновых векторов $k$. В [10] волны на поверхности цилиндрической ячейки возбуждали с помощью кольца, касающегося поверхности воды вблизи стенок ячейки. На поверхности возбуждалась только радиальная мода. В этом случае стоячие волны на поверхности описываются функцией Бесселя параметра $Rk$ ($R$ – радиус ячейки). Скаляр $k$ играет роль волнового числа: при больших значениях $R/\lambda$, ($\lambda = 2*\pi/k$ – длина возбуждаемой волны) и на большом расстоянии $r \gg \lambda$ от центра ячейки в узком угловом секторе цилиндрическую волну можно рассматривать как плоскую волну с волновым числом k в одномерном k-пространстве. Экспериментальные результаты [10] оказались в хорошем согласии с теорией слабой (волновой) турбулентности [1].

В настоящем сообщении представлены экспериментальные результаты исследований турбулентности в системе волн на поверхности воды, возбуждаемых вертикальными колебаниями ячейки за счет краевого эффекта смачивания при ускорениях меньше порового значения возникновения неустойчивости Фарадея в цилиндриче- ской ячейке, когда вихревое движение еще не наблюдается, и в квадратной ячейке, где вихревое движение при этих уровнях накачки хорошо развито.

Напомним, что в турбулентном каскаде можно выделить три диапазона частот: область накачки, в которой в систему поступает энергия; инерционный интервал, где энергия передается в основном за счет нелинейного взаимодействия; и область диссипации, в которой энергия колебаний переходит в тепло. В инерционном интервале парная корреляционная функция отклонений поверхности от равновесия $I_\omega$ описывается степенной функцией:
\begin{equation}
% \label{eq:disper}
I_\omega = \omega^{-m},
\end{equation}
где $\omega = 2 pi f$ – частота, показатель степени m зависит от спектральной характеристики возбуждающей силы [1]: m = –17/6 для широкополосной накачки и m = –23/6 для узкополосной накачки. Теоретические оценки m были подтверждены в экспериментах с жидким водородом [11] и с водой [2].

Край инерционного интервала определяется как характерная частота fb, при которой нелиней- ное время взаимодействия волн сравнивается со временем вязкого затухания [10]. Положение частоты $f_b$ зависит как от свойств поверхности жидкости, так и от характеристик накачки: амплитуды, ширины полосы возбуждения. Теория пред- сказывает степенную зависимость положения края инерционного интервала $\omega_b$ от амплитуды накачки A:
\begin{equation}
% \label{eq:disper}
f_b = A^{\beta}.
\end{equation}

Показатель степени $\beta$ зависит от типа накачки: при монохроматической накачке $\beta = 4/3$, при широкополосной – $\beta = 12/5$ [11]. Впервые край инерционного интервала в системе капиллярных волн наблюдался на поверхности жидкого водорода [10].
Диссипативную область можно охарактеризовать двумя параметрами: положением края инерционного интервала $f_b$ и характерной частотой экспоненциального затухания $f_d$. На частотах $f > f_b$ степенной закон распределения $I_\omega$ переходит в экспоненциальное затухание [11]:
\begin{equation}
% \label{eq:disper}
I_\omega = exp(f/f_b).
\end{equation}


Экспоненциальное падение в турбулентном каскаде на частотах выше $f_b$ наблюдали в системе капиллярно-гравитационных волн на поверхности жидкого водорода и гелия [12, 13]. В экспериментах с жидким водородом [12] в случае широкополосной накачки было показано, что характерная частота $f_d$ растет с увеличением амплитуды возбуждающей силы по степенному закону $f_d \sim A^{0.85}$.

Экспериментальные ячейки имеют конечные размеры, поэтому спектр поверхностных возбуждений носит дискретный характер. Это наклады- вает дополнительные ограничения на выполне- ние законов сохранения энергии и импульса [14]. В экспериментальных работах [15, 16] было пока- зано, что выбором размеров ячейки при накачке на некоторых частотах можно организовать пере- дачу энергии как на высокие, так и на низкие частоты.
Целью настоящей работы было проведение подробных исследований зависимостей высокочастотного края инерционного интервала $f_b$ и характерной частоты $f_d$ от амплитуды возбуждающей силы на поверхности воды в цилиндрической и квадратной ячейках при амплитудах накачки меньше порогового значения, при котором возникает параметрическая неустойчивость Фарадея.



\section{Экспериментальная методидка} \label{sect2_1}

Экспериментальная установка, схематично представленная на рис. 1, состоит из виброплатформы 1, установленной на ней экспериментальной ячейки с водой 2 и системы регистрации колебаний 3, 4.

Экспериментальные ячейки имели форму стакана диаметром от 65 до 130 мм и глубиной 10 мм, а также прямоугольника со сторонами 49 $\times$ 50 мм и глубиной 10 мм. Вода наливается выше края стенок стакана так, чтобы образовался выгнутый мениск. При вертикальных осцилляциях ячейки равновесный радиус мениска меняется в зависимости от ускорения ячейки, благодаря чему возбуждаются колебания поверхности воды. Волны на поверхности воды возбуждали, подавая переменное электрическое напряжение с цифрового генератора на вход виброплатформы. Использовали следующие виды накачек: монохроматическую на резонансной частоте, узкополосную с шириной полосы около 1 Гц и широкополосную 30–50 Гц. Под амплитудой накачки А в случае монохроматического возбуждения понимается амплитуда электрического сигнала, подаваемого на виброплатформу. В случае узкополосной или широкополосной накачки за амплитуду А принимается среднеквадратичное значение электрического сигнала, подаваемого на виброплатформу. Отметим, что высота волны основной гармоники на поверхности воды прямо пропорциональна амплитуде А (ускорению ячейки в вертикальном направлении) при монохроматической накачке.

Для регистрации колебаний поверхности воды была использована система, ранее описанная в [17]. Скользящий под небольшим углом лазерный луч падает на поверхность воды и отражается от нее. Мощность отраженного лазерного луча зависит от угла отражения. Поэтому присутствие волн на поверхности приводит к временным вариациям мощности отраженного луча $P(t)$. Отраженный луч фокусируется на фотодетектор, электрический сигнал с которого оцифровывается и записывается в память компьютера.

Регистрация волн на поверхности воды происходит в режиме “широкого луча”, т.е. характерная длина волн на поверхности воды много меньше размера пятна лазерного луча. В этом режиме мощность, регистрируемая фотодетектором, является интегральной характеристикой формы поверхности. В [17] показано, что при равномерном распределении световой мощности по лазерному пятну на поверхности жидкости парная корреляционная функция $I_\omega$ прямо пропорциональна квадрату компоненты Фурье мощности $P_\omega^2$ отраженного лазерного луча:
\begin{equation}
% \label{eq:disper}
I_\omega = P_\omega^2.
\end{equation}
На рис. 2 показано экспериментальное распределение амплитуд волн по частоте при возбуждении волн в ячейке диаметром 65 мм, глубиной 10 мм монохроматической силой на фиксированной частоте накачки $f_p$?. Частота накачки увеличивается от 46 до 90 Гц с шагом 0.1 Гц. По оси абсцисс отложена частота в герцах, а по оси ординат номер регистрируемой гармоники $N$ с частотой $N f_p$.
Оценка показывает, что в интервале частот от 45 до 90 Гц расстояние между резонансными пи- ками превосходит ширину пиков, вязкое уширение резонансных пиков $2\nu\omega^{4/3}(\rho/\sigma)^{2/3}$ ($\rho$ – плотность, $\nu$ – кинематическая вязкость, $\sigma$ – коэффициент поверхностного натяжения воды), т.е. спектр поверхностных колебаний в этом диапазоне частот является дискретным.
На рис. 3 приведен пример записи сигнала, регистрируемого фотодетектором при монохроматическом возбуждении волн на частоте 46 Гц в цилиндрической ячейке. Отметим, что основные вариации мощности отраженного лазерного луча обусловлены колебаниями поверхности воды на частоте накачки.


\section{Экспериментальные результаты} \label{sect2_2}

На рис. 4 приведен спектр $P^2_\omega$ , полученный фурье-преобразованием сигнала, показанного на рис. 3. Отметим особенности на этом распределении. Самый большой пик, расположенный слева на шкале частот, соответствует частоте возбуждающей монохроматической силы $f_b = 46$ Гц. Стрелкой на спектре отмечен край инерционного интервала $f_b$. На частотах выше $f_b$ располагается диссипативная область, в которой турбулентный поток энергии быстро затухает. В диапазоне между областью накачки и краем инерционного интервала располагаются пики, соответствующие резонансам, возникшим в результате трехволнового взаимодействия c частотами, кратными $f_b$. Видно, что максимумы пиков в спектре в пределах инерционного интервала хорошо ложатся на прямую линию, которая соответствует степенному закону распределения с показателем степени, близким к –3.

С увеличением амплитуды возбуждающей силы положение высокочастотного края инерционного интервала $f_b$ сдвигается в сторону высоких частот. Частота $f_b$ определяется по компенсированным спектрам $P_\omega^2/\omega^\gamma$. Показатель степени $gamma$ подбирается так, чтобы в инерционным интервале спектр $P_\omega^2/\omega^\gamma$ не зависел от частоты. Значение $f_b$ определяется как частота, при которой отклонение $P_\omega^2/\omega^\gamma$ от плоского спектра составляет 50%

\section{Высокочастотный край инерционного интервала} \label{sect2_3}

Анализ зависимостей $f_b(А)$ показывает, что показатель степени $\beta$ зависит от амплитуды возбуждающей силы. На рис. 5 приведено распределение волн по частотам и амплитудам, полученное при двух последовательных циклах увеличения и уменьшения амплитуды возбуждающей силы на частоте накачки, равной $f_p = 44$ Гц. Край инерционного интервала располагается на границе серого и черного цветов и возрастает по мере повышения амплитуды накачки. На рисунке хорошо видно, что при малых амплитудах накачки частота высокочастотного края инерционного интервала растет по закону немного сильнее линейного, т.е. $\beta > 1$ вплоть до 80-й гармоники частоты накачки. При больших амплитудах накачки показатель степени $\beta$ приближается к единице.

На рис. 6 в логарифмическом масштабе приведены зависимости частоты края инерционного интервала $f_b$ от амплитуды монохроматической накачки, полученные в цилиндрической ячейке диаметром 65 мм при накачке на частоте 45.5 Гц, в цилиндрической ячейке диаметром 130 мм при накачке на частоте 44 Гц. Видно, что полученные точки $f_b$ на графиках в логарифмических координатах хорошо ложатся на прямую линию, что соответствует степенной зависимости частоты края инерционного интервала от амплитуды накачки $f_b \sim А^\beta$. Экспериментально полученные значения показателя степени $\beta$ лежат в интервале от $1.3 \pm 0.2$ для всех цилиндрических ячеек. Отметим, что тео ретическое значение $\beta$ равно 4/3 для монохроматической накачки [13]. Таким образом, можно заключить, что экспериментальные значения находятся в хорошем согласии с теоретической оценкой.

При переходе от монохроматической к широкополосной накачке также наблюдаются хорошие степенные зависимости $f_b \sim А^\beta$ в широком интервале амплитуд возбуждающей силы, но показатель степени $\beta$ меньше, чем в случае монохроматической накачки.

На рис. 7 приведены в логарифмическом масштабе зависимости частоты края инерционного интервала от амплитуды широкополосной накачки для двух ячеек. Видно, что экспериментальные точки хорошо описываются степенными функциями амплитуды. Показатель степени $\beta$ изменяется от 1.32 в ячейке диаметром 65 мм до 1.14 в ячейке диаметром 130 мм, в среднем $\beta = 1.23 \pm 0.09$.

Также как и в случае монохроматической накачки экспериментальные точки на графиках хорошо ложатся на прямую линию, что соответствует степенной зависимости частоты края инерционного интервала от амплитуды накачки. Отметим, что полученные величины показателя степени $beta$ составляют в среднем $1.10 \pm 0.15$ и значительно отличаются от теоретического значения, равного 12/5 [11].
Чтобы убедиться, что полученные степенные зависимости $f_b(А)$ не являются особенностью волн в цилиндрической 
ячейке, эксперименты были повторены на почти квадратной ячейке со сторонами $49 \times 50$ мм. На рис. 8 приведены зависимости частоты высокочастотного края инерционного интервала $f_b$ от амплитуды монохроматической и широкополосной возбуждающей силы. Видно, что в этих случаях зависимость $f_b(А)$ можно также описать степенной функцией с показателем степени, равным $0.95 \pm 0.03$. Оказалось, что присутствие вихрей на поверхности не оказывает существенного влияния на волновую турбулентность.

Отметим, что при узкополосной накачке (ширина полосы ~1 Гц) в полученных спектрах $P^2_\omega$ край инерционного интервала слабо выражен, вследствие чего зависимость края инерционного интервала от амплитуды накачки установить не удается.

\section{Характреная частота затухания} \label{sect2_3}
Как отмечалось выше, на частотах выше $f_b$ турбулентный каскад затухает в результате вязких потерь. На рис. 9 в двойных логарифмических координатах приведены зависимости характерной частоты $f_d$ от амплитуды монохроматической накачки для двух ячеек. Отметим, что в ячейке диаметром 65 мм (рис. 9а) частота $f_d$ возрастает по степенному закону при повышении амплитуды возбуждающей силы до значения А = 12. В ячейке большего диаметра наблюдается монотонное повы- шение характерной частоты с ростом уровня накачки. Приближение экспериментальных данных степенной зависимостью $f_d \sim А^\alpha$ дает следующее значения показателя степени $\alpha = 1.18$ для ячейки диаметром 65 мм на растущем участке и $\alpha = 1.38$ для ячейки диаметром 130 мм, в среднем $\alpha = 1.28 \pm 0.10$.

На рис. 10 показано распределение $P^2_\omega$ в полулогарифмических координатах при широкополосной накачке в диапазоне 30 $\times$ 50 Гц, полученное в ячейке диаметром 65 мм. Амплитуды накачки в относительных единицах составляют 3:7:10. Стрелками на каждом спектре показано положение края инерционного интервала $f_b$, которое составляет 650, 2500 и 4300 Гц соответственно. На частотах выше $f_b$ в диссипативной области наблюдается экспоненциальное затухание. Для наглядности на рисунке проведены прямые линии, подчеркивающие экспоненциальные зависимости $P^2_\omega$. Характерные частоты, рассчитанные по зависимостям (2) для распределений, показанных на рис. 8, составляют 430, 1050, 1550 Гц соответственно.

В случае широкополосной накачки зависимость частоты $f_d$ от амплитуды накачки является монотонной. На рис. 11 приведены в логарифмическом масштабе зависимости $f_d$ от амплитуды широкополосной накачки в двух ячейках. Видно, что точки на графике хорошо ложатся на прямую линию, соответствующую степенной зависимости от амплитуды с показателем степени, близким к значениям, полученным по зависимостям частоты края инерционного интервала от амплитуды накачки. Полученные значения $\alpha$ близки к 1.1.

\section{Обсуждение} \label{sect2_3}

Экспериментальные данные свидетельствуют, что амплитудные зависимости частоты высокочастотного края инерционного интервала $f_b$ можно хорошо описать степенными функциями амплитуды $A^\beta$. Показатель степени при монохроматиче- ской накачке составляет в среднем $\beta = 1.23 \pm 0.09$, что близко к теоретической оценке $\beta = 4/3$. Амплитудная зависимость характерной частоты экспоненциального затухания турбулентного каскада описывается степенной функцией с показателем, равным $\alpha = 1.28 \pm 0.10$ Поэтому можно утверждать, что в случае монохроматической накачки частота $f_b$ прямо пропорционально $f_d$, $f_b = m f_d$. Значение коэффициента пропорциональности m составляет 4–5, т.е. характерная частота $f_d$ в несколько раз меньше частоты края инерционного интервала.

При широкополосной накачке показатели степени составляют в среднем $\beta = 1.10 \pm 0.15$ и $\alpha = 1.12 \pm 0.03$, т.е. они практически совпадают. Поэтому можно полагать, что $f_b = n f_d$, а значение n составляет 3–4. Это означает, что гармоники из диссипативной области как при монохроматической накачке, так и при широкополосном возбуждении взаимодействуют, в основном, с модами, расположенными в пределах инерционного интервала [12], но ниже частоты его края. Как близко эти моды расположены к краю инерционного интервала, установить сложно, но они значительно выше частоты накачки $f_p$.
Отметим, что в экспериментах со сверхтекучим гелием при монохроматической накачке характерная частота $f_d$ была близка к частоте накачки $f_p$ [13] и меньше частоты края инерционного интервала в десятки раз. В то же время при широкополосной накачке в экспериментах с жидким водородом характерная частота $f_d$ была только в несколько раз ниже частоты высокочастотного края инерционного интервала $f_b$ и значительно превосходила частоту накачки $f_p$ [12].

Таким образом, можно полагать, что отношение частот $f_b$, $f_p$, $f_d$ в экспериментах на поверхности воды, как при широкополосной, так и монохроматической накачке и в экспериментах с широкополосной накачкой поверхности водорода качественно близки. Во всех этих случаях волны из диссипативной области взаимодействуют в основном с модами из инерционного интервала, вдали от областей накачки и края инерционного интервала.

Обратим внимание, что при высоких уровнях монохроматической накачки частота $f_b$ растет с повышением амплитуды по закону слабее линейного (рис. 5). Остается непонятным расхождение в величине показателя степени $\beta$, полученного в эксперименте и оцененного из теории при широкополосном возбуждении волн. Отметим, что это разногласие наблюдается в экспериментах как с волнами на поверхности воды при различных методиках возбуждения поверхности [18], так и на поверхности жидкого водорода [10] и сверхтекучего гелия [13]. Во всех этих экспериментах угловые амплитуды волн на частоте накачки одного порядка, что связано с особенностью методики регистрации отклонения поверхности жидкости от равновесного положения [17], а кинематическая вязкость жидкостей изменяется примерно в 100 раз. То есть расхождения в значениях $\beta$ не связаны со свойствами жидкостей, а определяются другими причинами. Для прояснения этого во- проса нужны дополнительные исследования.

\section{Выводы} \label{sect2_4}

В работе экспериментально показано, что при возбуждении турбулентного состояния на по-верхности воды монохроматической или широкополосной накачкой высокочастотный край инерционного интервала и характерная частота в диссипативной области отличаются в несколько раз и качественно одинаково повышаются с ростом амплитуды накачки по степенному закону с показателем степени, близким к теоретически оцененному значению для монохроматического возбуждения. В случае широкополосной накачки наблюдается значительное расхождение между экспериментальными и теоретически оцененными значениями показателя $\beta$.
%\newpage
%============================================================================================================================
%\section{Методы регистрации волн} \label{sect2_2}


%\newpage
%============================================================================================================================
%\section{Методика регистрации вихрей} \label{sect2_3}
%
%
%\section{Экспериментальные установки} \label{sect2_4}
%\subsection{Регистрация волн на поврехности жидкого водорода}
%\subsection{Регистрация капиллярных волн на поврехности воды}
%\subsection{Регистрация вихрей возбуждаемых системой капиллярных волн на поврехности воды}
%\subsection{Регистрация вихрей возбуждаемых системой гравитационных волн на поврехности воды}
