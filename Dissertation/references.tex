\clearpage                                  % В том числе гарантирует, что список литературы в оглавлении будет с правильным номером страницы
%\hypersetup{ urlcolor=black }               % Ссылки делаем чёрными
%\providecommand*{\BibDash}{}                % В стилях ugost2008 отключаем использование тире как разделителя 
\urlstyle{rm}                               % ссылки URL обычным шрифтом
\ifdefmacro{\microtypesetup}{\microtypesetup{protrusion=false}}{} % не рекомендуется применять пакет микротипографики к автоматически генерируемому списку литературы
%\insertbibliofull                           % Подключаем Bib-базы
\ifdefmacro{\microtypesetup}{\microtypesetup{protrusion=true}}{}
\urlstyle{tt}                               % возвращаем установки шрифта ссылок URL
%\hypersetup{ urlcolor={urlcolor} }          % Восстанавливаем цвет ссылок
%(pam-pam \cite{Brazhnikov_bound} taram)

%\chapter*{Публикации по теме диссертации}						% Заголовок
\addcontentsline{toc}{chapter}{Публикации по теме диссертации}	% Добавляем его в оглавление


[F1] Brazhnikov M.Yu., Abdurakhimov L.V., Filatov S.V., Levchenko A.A., "Quasi-Planck" scpectra of capillary turbulence on the surface of liquid hydrogen // JETP Lett. 2011.V. 93. P. 34.

[F2] Л.В. Абдурахимов, М.Ю. Бражников, А.А. Левченко, И.А. Ремизов, С.В. Филатов, "Турбулентный капиллярный каскад вблизи края инерционного интервала на поверхности квантовой жидкости", Письма в ЖЭТФ, том 95 вып. 12, с. 751-760 (2012)

[F3] Л.В. Абдурахимов, М.Ю. Бражников, А.А. Левченко, И.А. Ремизов, С.В. Филатов, "Кинетическая и дискретная турбулентность на поверхности квантовой жидкости", УФН, том 182, 8, с. 879 (2012)

[F4] С.В. Филатов, М.Ю. Бражников, А.А. Левченко, "Метод пространственной регистрации волн на поверхности прозрачной жидкости", ПТЭ, 1, с. 107-112, (2014)

[F5] С.В. Филатов, М.Ю. Бражников, А.А. Левченко, "Формирование вихревого течения волнами на поверхности жидкости", Письма в ЖЭТФ, том 102, вып. 7, с. 486-490 (2015)

[F6] S.V. Filatov, V.M. Parfenyev, S.S. Vergeles, M.Yu. Brazhnikov, A.A. Levchenko, V.V. Lebedev, "Nonlinear Generation of Vorticity by Surface Waves", Physical Review Letters, 116, 054501 (2016)

[F7]. С.В. Филатов, М.Ю. Бражников, А.А. Левченко,  А.М. Лихтер, "Турбулентность в системе капиллярных волн на поверхности воды", Поверхность, 10, с. 69-76 (2016)

[F8] С.В. Филатов, С.А. Алиев, А.А. Левченко, Д.А. Храмов, "Генерация вихрей гравитационными волнами на поверхности воды", Письма в ЖЭТФ, том 104, вып.  10, с. 714-720 (2016)


\begin{thebibliography}{99}

\bibitem{zakh}
V.E. Zakharov, V.S. L'vov, G. Falkovich, Kolmogorov Spectra of Turbulence I, Springer-Verlag, (1992).

\bibitem{2}
М.Ю. Бражников, Г.В. Колмаков, А.А. Левченко, Л.П. Межов-Деглин, Капиллярная турбулентность на поверхности жидкого водорода, Письма в ЖЭТФ, том 73, вып 8, стр 443-446 (2001).
 
\bibitem{3}
G.V. Kolmakov, A.A. Levchenko, M.Yu. Brazhnikov et al, Phys. Rev. Lett. 93, 074501 (2004).

\bibitem{4}
V.E. Zakharov, N.N. Filonenko, J. App. Mech. Tech. Phys. 8, 62, (1967)

\bibitem(5)
V.M. Malkin, JEPT 86, 1263 (1984).

\bibitem{6}
I.V. Ryzhenkova, G.E. Falkovich, JEPT 98, 1931 (1990).

\bibitem{Brazhnikov_IET}
A.A. Levchenko, M.Yu. Brazhnikov, L.P. Mezhov-Deglin. Excitation and Detection of Nonlinear Waves on a Charged Surface of Liquid Hydrogen. IET, 45(6), 758–763 (2002).

\bibitem{Brazhnikov_bound_freq}
М.Ю. Бражников, Г.В. Колмаков, А.А. Левченко, Л.П. Межов-Деглин, Изменение граничной частоты инерционного интервала тербулентности капиллярных волн на поверхности жидкого водорода. Письма в ЖЭТФ, 74(12), 660–663 (2001).
%-----------------------------------
\bibitem{9}
1. Zakharov V.E., L’vov V.S., Falkovich G. Kolmogorov Spectra of Turbulence I. Springer–Verlag, 1992.

\bibitem{10}
2. Falcon E., Laroche C., Fauve S. // Phys. Rev. Lett. 2007. V. 98. P. 094503.

\bibitem{11}
3. Henry E., Alstrшm P., Levinsen M.T. // Europhys. Lett. 2000. V. 52 (1). P. 27.

\bibitem{12}
4. Shats M., Punzmann H., Xia H. // Phys. Rev. Lett. 2010. V. 104. P. 104503.

\bibitem{13}
5. Denissenko P., Lukaschuk S., Nazarenko S. // Phys. Rev. Lett. 2007. V. 99. P. 014501.

\bibitem{14}
6. Shats M.G., Xia H., Punzmann H. // Phys. Rev. E. 2005. V. 71. P. 046409.

\bibitem{15}
7. Von Kameke A., Huhn F., Fernandez-Garcia G., Munuzuri A.P., Perez-Munuzuri V. // Phys. Rev. Lett. 2011. V. 107. P. 074502.

\bibitem{16}
8. Филатов С.В., Бражников М.Ю., Левченко А.А. // Письма в ЖЭТФ. 2015. Т. 102. № 7. С. 486.

\bibitem{17}
9. Filatov S.V., Parfenyev V.M., Vergeles S.S., Brazhnikov M.Yu., Levchenko A.A., Lebedev V.V. // Phys. Rev. Lett. 2016. V. 116. P. 054501.

\bibitem{Brazhnikov_liq_hydr}
10. Бражников М.Ю., Колмаков, Г.В., Левченко А.А., Турбулентность капиллярных волн на поверхности жидкого водорода // ЖЭТФ. 2002. V. 122. P. 521.

\bibitem{Ryzhenkova}
11. Ryzhenkova I.V., Falkovich G.E., Effect of dissipation on the structure of a stationary wave turbulence spectrum // JETP. 1990. V. 71. P. 1085.

%\bibitem{20}
%12. Brazhnikov M.Yu., Abdurakhimov L.V., Filatov S.V., Levchenko A.A., "Quasi-Planck" scpectra of capillary turbulence on the surface of liquid hydrogen // JETP Lett. 2011.V. 93. P. 34.

%\bibitem{Abdur_quantum}
%13. Абдурахимов Л.В., Бражников М.Ю., Левченко А.А., Ремизов И.А., Филатов С.В., Турбулентный капиллярный каскад %вблизи края инерционного интервала на поверхности квантовой жидкости // Письма в ЖЭТФ. 2012. Т. 95. С. 751.

\bibitem{22}
14. Kartashova E.A. // Physica D. 1991. V. 54. P. 125.

\bibitem{23}
15. Brazhnikov M.Yu., Levchenko A.A., Mezhov-Deglin L.P., Remizov I.A. // JETP Lett. 2014. V. 100. P. 669.

\bibitem{24}
16. Абдурахимов Л.В., Бражников М.Ю., Левченко А.А., Лихтер А.М., Ремизов И.А. // Физика низких температур. 2015. Т. 41. № 3. С. 215.

%\bibitem{25}
%17. Бражников M.Ю., Левченко А.А., Межов-Деглин Л.П. // Приборы и техника эксперимента. 2002. № 6. С. 31.

\bibitem{Brazhnikov_EPL}
M.Y. Brazhnikov, G.V. Kolmakov, A.A. Levchenko, L.P. Mezhov-Deglin, Observation of capillary turbulence on the water surface in a wide range of frequencies // Europhys. Lett. 58 (4), 510-516 (2002)

%%---------------------------------------------------

\bibitem{27}
1. T.H. Havelock, Phil, Mag. 8, 569 (1929).

\bibitem{28}
2. В.А. Калиниченко, С.В. Нестеров, Н.Л. Никитин, С.Я. Секерж-Зенькович, Изв. АН СССР, ФАО 4, 432 (1982).

\bibitem{29}
3. М.Ю. Бражников, А.А. Левченко, Л.П. Межов- Деглин, Приборы и техника эксперимента 45, 31 (2002) [Instr. Exp. Tech. 45, 758 (2002)].

\bibitem{30}
4. J. Miles and D. Henderson, Annu. Rev. Fluid Mech. 22, 143 (1990).

\bibitem{31}
5. M. Faraday, Phil. Trans. R. Soc. London 121, 299 (1831).

\bibitem{32}
6. R. Ramshankar, D. Berlin, and J.P. Gollub, Phys. Fluids A 2, 1955 (1990).

\bibitem{33}
7. A.vonKameke, F. Huhn, G. Fernandez-Garcia, A.P. Munuzuri, and V. Perez-Munuzuri, Phys. Rev. Lett. 107, 074502 (2011).

\bibitem{34}
8. N. Francois, H. Xia, H. Punzmann, S. Ramsden, and M. Shats, Phys. Rev. X 4, 021021 (2014).

\bibitem{35}
9. R. Kraichnan, Phys. Fluids 10, 1417 (1967).

\bibitem{36}
10. O.N. Mesquita, S. Kane, and J.P. Gollub, Phys. Rev. A 45, 3700 (1992).

\bibitem{37}
11. G.G. Stokes, Trans. Cambridge Phil. Soc. 8, 441 (1847).

%\bibitem{38}
%12. Л.Д. Ландау, Е.М. Лифшиц, Теоретическая физика. Т. 6. Гидродинамика. Физматлит, М. (2003).

\bibitem{PIVlab}
13. W. Thielicke and E. J. Stamhuis, J. Open Res. Soft. 2(1), е30 (2014).

\bibitem{Lukaschuk}
14. S. Lukaschuk, P. Denissenko, and G. Falkovich, Eur. Phys. J. Special Topics 145, 125 (2007).

%\bibitem{41}
%15. V.V. Lebedev, V.M. Parfenyev, and S. Vergeles, to be published.

%%-------------------
\bibitem{42}
[1] H. Lamb, Hydrodynamics, 6th ed. (Cambridge University Press, Cambridge, England, 1975).

\bibitem{43}
[2] L. D. Landau and E. M. Lifshitz, Course of Theoretical Physics, Vol. 6, Fluid Mechanics, 2nd English ed., (Pergamon Press, Oxford, England, 1987).

\bibitem{44}
[3] A. von Kameke, F. Huhn, G. Fernandez-Garcia, A. P. Munuzuri, and V. Perez-Munuzuri, Phys. Rev. Lett. 107, 074502 (2011).

\bibitem{Shats_inverse}
[4] N. Francois,H.Xia, H. Punzmann, and M. Shats, Phys. Rev. Lett. 110, 194501 (2013).

\bibitem{46}
[5] N. Francois, H. Xia, H. Punzmann, S. Ramsden, and M. Shats, Phys. Rev. X 4, 021021 (2014).

\bibitem{47}
[6] M. Faraday, Phil. Trans. R. Soc. London 121, 299 (1831).

\bibitem{48}
[7] See Supplemental Material at http://link.aps.org/
supplemental/10.1103/PhysRevLett.116.054501 for details of the calculations.

\bibitem{49}
[8] W. Thielicke and E. J. Stamhuis, J. Open Res. Software 2, e30 (2014).

\bibitem{Falkovich}
[9] G. Falkovich, J. Fluid Mech. 638, 1 (2009).

\bibitem{Buhler}
[10] O. Bühler and M. Holmes-Cerfon, J. Fluid Mech. 638,5 (2009).

\bibitem{Punzmann}
[11] H. Punzmann, N. Francois, H. Xia, G. Falkovich, and M. Shats, Nat. Phys. 10, 658 (2014).



\bibitem{parf}
S.V. Filatov, V.M. Parfenyev, S.S. Vergeles et al., Phys.Rev. Lett. { 116}. P. 054501 (2016).

\bibitem{kameke}
A. Von Kameke, F. Huhn, G. Fernandez-Garcia et al., Phys. Rev. Lett. { 107}. P. 074502 (2011).

\bibitem{shats}
N. Francois, H. Xia, H. Punzmann et al., Pys.Rev.Lett. { 110 }, 194501 (2013)

\bibitem{fil}
С.В. Филатов, М.Ю. Бражников, А.А. Левченко, Письма в ЖЭТФ. {f 102}. № 7. С. 486 (2015).

\bibitem{land}
Л.Д. Ландау, Е.М. Лифшиц, Теоретическая физика, т. 6, Гидродинамика. М.: Физматлит, (2003).

\bibitem{piv}
W. Thielicke and E.J. Stamhuis Journal of Open Research Software 2(1)p.e30 (2014).

\bibitem{fran}
N. Francois, H. Xia, H. Punzmann et al., Phys. Rev. X. { 4}, 021021 (2014)

\bibitem{krai}
R. H. Kraichnan, Phys. Fluids, {10}, 1417 (1967).


\end{thebibliography}
