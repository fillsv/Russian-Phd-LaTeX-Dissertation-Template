\addcontentsline{toc}{chapter}{Введение}
\chapter*{Введение}\label{intro}

\section{Волновая турбулентность}% \label{subsect1_turb}
Теория слабой волновой турбулентности \cite{Zakharov}  описывает многочисленные системы слабо взаимодействующих волн: рябь на воде и гравитационные волны на поверхности океана, волны Россби в атмосфере планет и в мировом океане, Ленгмюровские волны в плазме и спиновые волны в магнетиках.

Для возникновения турбулентности необходимым условием является наличие в динамической системе большого числа степеней свободы. В системе поверхностных волн  к степеням свободы можно отнести  волны с разными волновыми векторами. Причем длины волн изменяются от долей миллиметра до километра, то есть их отношение может превышать 9 порядков. Согласно теории слабой волновой турбулентности \cite{Zakharov}  при возбуждении системы на определенных масштабах волновых векторов энергия в силу нелинейного взаимодействия перераспределяется в k-пространстве (пространстве волновых векторов). Часть энергии уходит в малые масштабы (большие волновые вектора, прямой каскад), где диссипирует, а другая часть энергии передается в большие масштабы (обратный каскад), где также диссипирует в силу трения о дно и стенки сосуда. Причем специфической чертой  развитой турбулентности является наличие определенного диапазона масштабов, в котором доминирующим процессами не являются ни накопления, ни диссипации, а только передача энергии из одних масштабов в другие. 

На рисунке \ref{img:turb} показана схема развитого прямого турбулентного состояния. В этом турбулентном состоянии можно выделить три характерные области: область накачки, в которой энергия приходит в систему, инерционный интервал, где энергия передается практически без потерь и область диссипации, где энергия покидает систему. 

\begin{figure}[ht] 
  \center
  \includegraphics [scale=0.2] {Intro/iner_inter.jpg}
  \caption{} 
  \label{img:turb}  
\end{figure}

Наиболее наглядно волновая турбулентность проявляется в системе являются  волн на поверхности океанов и морей.

Ветер, дующий вдоль изначально гладкой поверхности   воды, осуществляет накачку энергии в систему волн из-за неустойчивости Кельвина-Гельмгольца \cite[c. 99]{NonLinearWaves}. При этом масштаб накачки составляет порядка одного сантиметра - капиллярная длина. В результате нелинейного взаимодействия образуются волны других масштабов, которые также эффективно поглощают энергию ветра - ветровые волны, и энергия передается как в сторону коротких волн, так и в сторону длинных волн. В результате нелинейных процессов на поверхности воды могут образоваться большие волны с характерной длиной волны в сотни метров и даже километры. 

Обратим внимание, что в стационарном турбулентном каскаде в системе волн осуществляется баланс энергии: сколько энергии приходит в систему, столько же и диссипирует в результате вязкого трения на малых масштабах, где вязкое затухание является доминирующим механизмом. Т.е. прямой каскад в волновой системе обеспечивает диссипацию энергии приходящей от внешнего источника. 


\section{Закон дисперсии волн на поверхности жидкости}% \label{subsect1_disper}

Волны на поверхности жидкости формируются за счет силы гравитации и сил поверхностного натяжения, причем для длинных волн преобладает влияние гравитации, а для коротких волн определяющими являются капиллярные силы. Это хорошо видно из закона дисперсии для поверхностных волн, однозначно связывающего угловую частоту волны $\omega$ и модуль волнового вектора $\mathbf{k}$ волн на свободной поверхности жидкости:
\begin{equation}
 \label{eq:disper_dip}
\omega^2 = (gk + \sigma/\rho k^3)th(kh),
\end{equation}
где $g$ - ускорение свободного падения, $\sigma$ - коэффициент поверхностного натяжения, $\rho$ - плотность жидкости, $h$ - глубина жидкости.

В случае, когда $k \gg (g\rho/\sigma)^{1/2}$ влияние гравитационных сил становится пренебрежимо малым по сравнению с капиллярными силами. Такие волны называют капиллярными. Волновые вектора $k \ll (g\rho/\sigma)^{1/2}$ соответствуют гравитационному участку закона дисперсии. В промежуточном случае, когда $k \sim (g\rho/\sigma)^{1/2}$, говорят о капиллярно-гравитационных волнах. Для свободной поверхности воды характерная частота перехода от гравитационных волн к капиллярным составляет $\sim$ 17 Гц, при этом длина волны равна $\lambda = 2\pi/k \approx$ 1.5 см. Для поверхности жидкого водорода эта частота также равняется $\sim$ 17 Гц, и ей соответствует длина волны $\sim$ 1.1 см.


Так как плотность жидкого водорода в 13 раз меньше плотности воды, то для возбуждения волн одинаковой амплитуды на поверхности жидкого водорода требуются значительно меньшие силы, чем для генерции волн на поверхности воды. 

В объем жидкового водорода можно ввести заряды различными методами. При приложении электрического поля перпендикулярно поверхности она заряжается. Затем, если в дополнение к постоянному поля добавить переменное, то можно возбуждать волны, воздействую электрическим полем непосредственно на заряженную поверхность. Это  приводят к уникальной возможности экспериментального изучения слабой волновой турбулентности \cite{Kolmakov2006}. Использование жидкого водорода в экспериментах по волновой турбулентности способствовало наблюдению явлений предсказанных волновой теорией, например стационарный спектр Захарова-Колмогорова в капиллярной турбулентности в широком диапазоне частот \cite{Brazhnikov2001}, а также позволило наблюдать новые явления, которые были успешно объяснены в рамках приближения теории слабой волновой турбулентности: например квазиадиабатический распад капиллярной турбулентности \cite{quasiadiabatic} и подавление высокочастотных турбулентных колебаний добавлением низкочастотной возбуждающей силы \cite{addLowFreq}.

Стоит отметить, что если глубина жидкости больше, чем характерная длина волны, то $kh\gg 1$ и $th(kh)$ можно считать равным 1. Таким образом, в приближении глубокой воды дисперсия гравитационно-капиллярных волн записывается как:


\begin{equation}
 \label{eq:disper}
\omega^2 = gk + \sigma/\rho k^3,
\end{equation}


В экспериментах с гравитационными волнами глубина жидкости была около $ h \approx 7$ см, минимальный волной вектор $k \approx 0.36$ см$^{-1}$, соответственно $th(kh) \approx 0.99$, т.е. влиянием глубины можно пренебречь.


В области высоких частот, где можно пренебречь влиянием гравитационных сил, закон дисперсии будет капиллярным:
\begin{equation}
 \label{eq:disperCap}
\omega^2 = \sigma/\rho k^3
\end{equation}

%В экспериментах с гравитационными волнами глубина жидкости была около $ h \sim 7$ см, минимальный волной вектор $k = 0.36$, соответсвенно $th(kh) \sim 0.99$, т.е. влиянием глубины можно также пренебречь. Таким образом закон дисперсии будет:
%\begin{equation}
% \label{eq:disperGrav}
%\omega^2 = gk,
%\end{equation}

\section{Законы сохранения энергии и импульса} %\label{subsect1_lawSave}

%Из \cite{Brazhnikov2001}. ""Известно, что капиллярные волны на поверхности жидкости характерезуются относительно сильным взаимодействием[?]. Ансамбль взаимодействующих волн может быть описан в рамках кинетического уравнения, вполне аналогичного уравнению Больцмана газовой динамики. Основной вклад в перераспределение энергии в обратном пространстве вносит трехволновые процессы - распад волны на две с сохранением суммарного волнового вектора и суммарной частоты, а так же обратный ему процесс слияния двух волн в одну.
При взаимодействии капиллярных волн выполняются законы сохранения энергии и импульса:
\begin{equation}
 \label{eq:saveOmega}
\omega_1 = \omega_2 \pm \omega_3
\end{equation}
\begin{equation}
 \label{eq:saveK}
\mbox{\boldmath$k_1$} = \mbox{\boldmath$k_2$} \pm \mbox{\boldmath$k_3$}
\end{equation}

Отметим, что если в законы дисперсии волн $\omega \sim k ^ \alpha$ показатель степени $\alpha > 1$, то трехволновые процессы распада-слияния волн удовлетворяют законам сохранения импульса и энергии. Такой закон дисперсии называют распадным. Если $\alpha < 1$ , трехволновые процессы запрещены, и основным взаимодействием волн являются четырех волновые процессы. В системе гравитационных волн трехволновые процессы запрещены, и основными являются четырехволновые. 

Таким образом, при возбуждении на поверхности жидкости волн в области может быть сформировано турбулентное состояние, в котором поток энергии направлен из области низких частот(область накачки) в сторону высоких частот. Теория слабой волновой турбулентности \cite{Zakharov} предсказывает, что основной вклад в перенос энергии по турбулентному капиллярному каскаду вносят как раз трехволновые процессы слияния волн. 


\section{Инерционный интервал турбулентного каскада}% \label{subsect1_1_3}

Как было сказано выше характерной особенностью турбулентного каскада является наличие инерционного интервала -  частотного диапазона в котором энергия, в основном, передается в k-пространстве из одного масштаба в другой. 
В настоящий момент имеется довольно много теоретических и экспериментальных работ посвященных изучению распределения энергии в инерционном интервале турбулентного каскада в различных системах. Теория слабой волновой турбулентности предсказывает степенное распределение энергии по шкале частот \cite{Zakharov}:
\begin{equation}
\label{eq:EOmega}
E_\omega \sim \omega^{-\alpha}
\end{equation}

С экспериментальной точки зрения удобно исследовать не распределение энергии по волновым векторам(или частотам), а парную корреляционную функцию отклонения поверхности от положения равновесия $I(\tau)=<\eta(r, t+\tau)\eta(r,t)>$, так как величину отклонения поверхности от положения равновесия, в отличии от энергии, можно непосредственно измерить в эксперименте. Фурье образ парной корреляционной функции отклонения поверхности от равновесного состояния связан с распределением энергии по частотам формулой:
\begin{equation}
\label{eq:EOmegaI}
I_\omega \sim E_\omega \omega^{-4/3} = n(\omega) \omega^{-1/3},
\end{equation}
где $n(\omega)$ - функция распределения капиллярных волн.

Таким образом предсказывается степенная зависимость парной корреляционной функции от частоты $I_\omega \sim \omega^{-m}$ в инерционном интервале.

В зависимости от характера накачки теория волновой турбулентности предсказывает различные значения показатели $m$. Для широкополосной накачки (когда ширина полосы накачки сопоставима с  частотой накачки), предсказывается $m$ = 17/6. При накачке узкополосным сигналом в спектре появляются равноудаленные пики, максимумы которых убывают c ростом частоты с показателем $m$ = 23/6. Данные предсказания подтверждаются как компьютерным моделированием \cite{Babiano1995, Babiano1987, Falcovich1988, Pushkarev1996}, так и экспериментальными исследованиями распределения $I_\omega$
в спектрах турбулентных каскадов в системе волн на поверхности воды \cite{BrazhnikovWater}, жидкого водорода \cite{Brazhnikov2001}, жидкого гелия \cite{Abdurakhimov2007}, ртути \cite{Falcon2007}.

Стоит отметить, что одной из сложностей экспериментального исследования турбулентных каскадов является степенное уменьшения энергии волны с ростом частоты. Так как величина показателя степени $m$ находится в районе 2-3, а диапазон частот, в котором существует турбулентный каскад, может достигать нескольких декад, то для экспериментального исследования поведения турбулентной системы в достаточно широком диапазоне частот, необходимым для наблюдения развитого турбулентного каскада, требуется экспериментальное оборудование обладающее большим динамическим диапазоном. По этой причине экспериментальные исследования были практически невозможны до появления широко распространенных АЦП с высоким динамическим диапазоном и достаточно высокой частотой оцифровки, а также компьютерной техники и программного обеспечения для обработки полученных сигналов.

\section{Положение высокочастотной границы инерционного интервала}\label{subsect_hiFreqBound}

При определении частотной области, где заканчивается инерционный интервал и начинается диссипация энергии важную роль играют такие характеристики волновой системы как вязкое время затухния волны и нелинейное время взаимодействия волн.

Время вязкого затухания волны на частоте $\omega$ \cite[стр. 135]{land}.
\begin{equation}
\label{eq:tauNu}
1/\tau_\nu = 2\nu k^2 = 2 \nu \omega^{4/3}(\sigma/\rho)^{2/3},
\end{equation}
где $\nu$ - кинематическая вязкость жидкости.
То есть вязкое время уменьшается с ростом частоты, а следовательно и диссипация энергии на более высоких частотах сильнее.


Характерное время нелинейного взаимодействия капиллярных волн можно выразить через параметры жидкости и функцию распределения капиллярных волн $n(\omega)$:
\begin{equation}
\label{eq:tauNl}
1/\tau_{nl} = |V_\omega|^2 n(\omega)
\end{equation}

Где $V(\omega) \approx (\sigma/\rho^{3/2})\omega^{3/2}$ - коэффициент трехволнового нелинейного взаимодействия капиллярных волн.

%Функцию распределения капиллярных волн $n(\omega) \sim E_{\omega}\omega^{-1}$ 
В развитом турбулентном каскаде энергия передается от низких частот к высоким практически без потерь до тех пор, пока поток энергии волн, переходящий в тепло, не становится сравнимым с потоком энергии по каскаду. Таким образом, положение высокочастотной границы инерционного интервала можно определить как частоту, на которой совпадают времена вязкого затухания и нелинейного взаимодействия волн.

Из уравнений (\ref{eq:EOmegaI}, \ref{eq:tauNu}, \ref{eq:tauNl}), используя известные значения $m$ = 17/6 для широкополосной накачки и $m$ = 23/6 для монохроматической накачки, получаем оценку амплитудной зависимости частоты границы инерционного интервала $\omega_b \sim \eta_p^{12/5}$ для широкополосной накачки и $\omega_b \sim \eta_p^{4/3}$ для узкополосной накачки. Отметим, работу \cite{Brazhnikov_bound_freq}, в которой изучалось поведения положения границы инерционного интервала при вариациях частоты и амплитуды монохроматической накачки в цилиндрической геометрии. Работ направленных на понимание общей картины поведения положения границы инерционного интервала при различных типах накачки в  экспериментальных  ячейках разной геометрии, не было.
%хренности $\Omega_0$ в случае стоячих волн определяется как:

\section{Диссипативная область турбулентного каскада}% \label{subsect_disp}

На высоких частотах, выше границы инерционного интервала, распределение $I_\omega$ определяется спектральной характеристикой накачки, нелинейным взаимодействием и затуханием волн. В диссипативной области время вязкого затухания волн не превышает время нелинейного взаимодействия: преобладают процессы затухания. Однако нелинейные процессы существенно влияют на форму спектра. Если волны в диссипативном интервале взаимодействуют в основном с ближайшими соседями, а не с волнами из инерционного интервала, то распределение энергии по волнам в области диссипации становиться близким к экспоненциальному.
Если же волны в диссипативной области $\omega \gg \omega_b$ взаимодействует волнами из инерционного интервала $\omega \ll \omega_b$, то распределение энергии по волнам несколько отличается от экспоненциального. Детальное рассмотрение дает “квазипланковский” спектр корреляционной функции в диссипативной области  \cite{Ryzhenkova1990}
\begin{equation}
% \label{eq:tauNu}
<\eta_\omega^2> \sim \omega^{-s} e^{-\omega/\omega_d},
\end{equation}			
где s - некая константа. Численные вычисления для дискретного кинетического уравнения \cite{Ryzhenkova1990} подтверждают экспоненциальную зависимость волнового числа заполнения в области сильного затухания. Величину $\omega_d$ имеющую размерность частоты и отвечающей за то насколько быстро затухает турбулентный спектр в диссипативной области будем называть частотой вязкого затухания спектра в диссипативной области.
	
%	Мы представляем первое экспериментальное наблюдение турбулентных спектров капиллярных волн в диссипативной области возбужденных низкочастотной случайной силой на поверхности жидкого водорода.

%
%Слабая волновая теория предсказывает существование стационарного неравновесного состояния в системе взаимодействующих капиллярных волн - спектр Колмогорова-Захарова для спектральной плотности энергии $\varepsilon(k)$ волн в инерционном интервале.
%\begin{equation}
% \label{eq:epsK}
% \varepsilon(k) \sim P^{1/2}k^{-7/4}
%\end{equation}
%%
%\todo{Разобраться с этим.(это нужно на стр. 28, ссылка на формулу(3))}
%
%Используя соотношение между спектральной плотностью энергии $\varepsilon(k)$ и парной корреляционной функцией высоты поверхности $\eta(\mathbf{r}, t)$, и принимая во внимание закон дисперсии капиллярных волн $\omega(k) \sim k^{3/2}$, можно получить частотный спектр корреляционной функции $<\eta(t+\tau)\eta(t)>$:
%\begin{equation}
% \label{eq:corrFun}
%<\eta_\omega^2> \sim (\sigma k^2)^{-1} \varepsilon(k)(d\omega/dk)^{-1} \sim P^{1/2} \omega^{-17/6}.
%\end{equation}
%
%Этот стационарный спектр характеризуется одной величиной P - потоком энергии на большие частоты. На больших частотах передаваемая энергия диссипирует из-за вязких потерь и турбулентный каскад разрушается. Следовательно для поддержания распределения (2) или (1) постоянным во времени, система должна постоянно накачиваться энергией на низких частотах. Высокочастотная граница инерционного интервала $\omega_d$, где распределение (2) еще остается верным, может быть оценена из предположения, что время вязкого затухания $\tau_\eta(\omega)$ и время нелинейного взаимодействия $\tau_{nl}(\omega)$ равны по порядку для волн на частоте $\omega_d$. Это предположение дает нам [5]:
%\begin{equation}
% \label{eq:omegaD}
%\omega_d \sim \bigg(\frac{P^{1/2}}{\eta}\bigg)^{6/5}\sim \bigg(\frac{\eta^2_0 \omega^{17/5}_0}{\eta} \bigg)^{6/5}
%\end{equation}


%Слабая волновая теория предсказывает существование стационарного неравновесного состояния в системе взаимодействующих капиллярных волн - спектр Колмогорова-Захарова для спектральной плотности энергии е(к) волн в инерционном интервале.
%\varepsilon(k) \sim P^{1/2} k^{-7/4}
%Используя соотношение между спектральной плотностью энергии и парной корреляционной функцией высоты поверхности (r, t), и принимая во внимание закон дисперсии капиллярных волн (k) k3/2, можно получить частотный спектр корреляционной функции <(t+)(t)>:
%<|2|>(k2)-1(k)(d/dk)-1P1/2-17/6
%	Этот стационарный спектр характеризуется одной величиной P - потоком энергии на большие частоты . На больших частотах передаваемая энергия диссипирует из-за вязких потерь и турбулентный каскад разрушается. Следовательно для поддержания распределения (2) или (1) постоянным во времени, система должна постоянно накачиваться энергией на низких частотах. Высокочастотная граница инерционного интервала d, где распределение (2) еще остается верным, может быть оценена из предположения, что время вязкого затухания () и время нелинейного взаимодействия $\tau_{nl}(\omega)$ равны по порядку для волн на частоте d. Это предположение дает нам [5]:
%<........>  (3)





%Экспоненциальное падение в турбулентном каскаде на частотах выше $\omega_b$ наблюдали в системе капиллярно-гравитационных волн на поверхности жидкого водорода и гелия [12, 13]. В экспериментах с жидким водородом [12] в случае широкополосной накачки было показано, что характерная частота $f_d$ растет с увеличением амплитуды возбуждающей силы по степенному закону $f_d \sim A^{0.85}$.

Экспериментальные ячейки имеют конечные размеры, поэтому спектр поверхностных возбуждений носит дискретный характер. Это накладывает дополнительные ограничения на выполнение законов сохранения энергии и импульса \cite{Kartashova1991}. В экспериментальных работах \cite{Brazhnikov2014, Aburakhimov2015} было показано, что выбором размеров ячейки при накачке на выбранных частотах можно организовать передачу энергии как на высокие, так и на низкие частоты.
Одной из целей настоящей работы было проведение подробных исследований зависимостей высокочастотного края инерционного интервала $\omega_b$ и характерной частоты $\omega_d$ от амплитуды возбуждающей силы на поверхности воды в цилиндрической и квадратной ячейках при амплитудах накачки меньше порогового значения, при котором возникает параметрическая неустойчивость Фарадея.

\section{Дискретные моды колебаний поверхности жидкости  в ячейке конечных размеров} %\label{subsect1_geometr}

Распределение волн на поверхности жидкости сильно зависит от геометрии сосуда и от способа возбуждения их возбуждения.
В случае, если затухание волн мало, при их возбуждении на поверхности жидкости возникнет система стоячих волн. Форма стоячих волн будет зависеть от граничных условий и возбуждаемой моды. Граничным условием для волн в ячейке конечных размеров  является неспособность воды проходить через стенку ячейки, т.е. нормальная (к стенке ячейки) компонента скорости жидкости должна быть равна нулю. Для цилиндрической ячейки радиуса $r_0$ резонансные моды волн будут описываться функцией Бесселя:

\begin{equation}
 \label{eq:Bessel}
h(r, \phi, t) = A J_0(k_nr) cos(m\phi) cos(\omega t),
\end{equation}
причем $k_n$ должна удовлетворять требованию ${J_0}'(k_nr_0) = 0$. Иными словами на границе ячейки должна быть пучность стоячей волны.

Если $m$ = 0, то мода будет радиальной, в таком случае скаляр $k$ играет роль волнового числа: при больших значениях $R/\lambda$, ($\lambda = 2\pi/k$ – длина возбуждаемой волны) и на большом расстоянии $r \gg \lambda$ от центра ячейки в узком угловом секторе цилиндрическую волну можно рассматривать как плоскую волну с волновым числом $k$ в одномерном k-пространстве.

В прямоугольной ячейке стоячие волны описываются суммой двух стоячих перпендикулярных синусоидальных волн:
\begin{equation}
\label{eq:waveStand}
h(x, y, t) = A_1 sin(kx)cos(\omega t)+A_2 sin(ky)cos(\omega t+ \phi)
\end{equation}
где $\phi$ разность фаз между стоячими волнами в разных направлениях.

Бегущие волны, распространяющиеся от двух перпендикулярных стенок прямоугольной ячейки будут задаваться выражением:
\begin{equation}
\label{eq:waveRun}
h(x, y, t) = A_1 sin(kx-\omega t)+A_2 sin(ky-\omega t+ \phi)
\end{equation}



\section{Возбуждение волн в ячейке, совершающей колебания в вертикальном направлении} \label{p1_methodsExt}
Турбулентноcть на поверхности воды в гравитационно-капиллярном интервале частот изучалась многими исследователями в течение нескольких последних десятилетий \cite{Falcon2007, Henry2000, Shats2010, Denissenko2007}. Для возбуждения волн использовали различные методики. В лабораторных условиях волны на поверхности жидкости могут возбуждаться различными способами: при помощи волнопродукторов \cite{Havelock1929, Falcon2007}, электрическими силами, действующими на границу раздела жидкостей с разной диэлектрической проницаемостью \cite{Kalinichenko1982} или на поверхность заряженной жидкости \cite{Brazhnikov2002}. Однако в большинстве работ для генерации волн используют параметрическую неустойчивость поверхности жидкости, совершающей вынужденные вертикальные колебания с ускорениями выше некоторого порогового значения (неустойчивость Фарадея) \cite{Henry2000, Shats2010, Denissenko2007}. Отличительной чертой этой методики является высокий уровень возбуждения волн сразу после возникновения неустойчивости на поверхности. Такая особенность методики возбуждения не позволяет работать с волнами малой амплитуды. Кроме того, как выяснилось, при сильном возбуждении наряду с нелинейным взаимодействием волн наблюдается генерация вихревого движения \cite{VonKameke2011, Francois2013}. Недавно в \cite{F5, F6} было показано, что завихренность формируется в результате взаимодействия нелинейных волн, имеющих неколлинеарные волновые векторы $\mathbf{k}$. В \cite{BrazhnikovWater} волны на поверхности цилиндрической ячейки возбуждали с помощью кольца, касающегося поверхности воды вблизи стенок ячейки. На поверхности возбуждалась только радиальная мода. В этом случае стоячие волны на поверхности описываются функцией Бесселя параметра $Rk$ ($R$ – радиус ячейки). Скаляр $k$ играет роль волнового числа: при больших значениях $R/\lambda$, ($\lambda = 2\pi/k$ – длина возбуждаемой волны) и на большом расстоянии $r \gg \lambda$ от центра ячейки в узком угловом секторе цилиндрическую волну можно рассматривать как плоскую волну с волновым числом k в одномерном k-пространстве. Экспериментальные результаты \cite{BrazhnikovWater} оказались в хорошем согласии с теорией слабой (волновой) турбулентности \cite{Zakharov}.


При вертикальных колебаниях ячейки возможны два механизма возбуждения волн на поверхности воды. Первый осуществляется благодаря наличию мениска на границе ячейки. При вертикальных колебаниях равновесный радиус мениска меняется в зависимости от величины вертикального ускорения, что приводит к появлению на поверхности жидкости волн с частотой равной частоте вертикальных колебаний ячейки. 

Второй механизм возникновения волн обусловлен развитием параметрической неустойчивости на поверхности жидкости, впервые описанной Фарадеем \cite{Faraday1831}. Неустойчивость Фараде является параметрической, а частота возбуждаемой волны оказывается в два раза меньше частоты вертикальных колебаний ячейки. 

При вертикальных колебаниях сосуда с относительно низкой амплитудой параметрическая неустойчивость Фарадея не возникает, так как неустойчивость развивается при амплитудах выше некоторого порогового значения. Поэтому для наблюдения спектров турбулентного каскада на поверхности воды использовалось возбуждение волн с помощью вертикальных колебаний с амплитудой ниже порога параметрической неустойчивости. 

Стоит отметить, что при возбуждении волн на поверхности воды с помощью мениска в цилиндрической ячейки возбуждаются только радиальные моды, а при развитии параметрической неустойчивости происходит возбуждение и азимутальной моды.



При наблюдении волновой турбулентности на поверхности жидкого водорода удобнее использовать возбуждение волн с помощью электрического поля. В этом случае амплитуда волн ограничивается напряжением пробоя и размерами оптического окна криостата, через которое с помощью лазерного луча производится регистрация волн \cite{Brazhnikov2002}.


\section{Метод детектирования волн на поверхности жидкости}\label{p1_methodDetect}

Существует несколько методик регистрации капиллярных волн. Можно регистрировать волны на поверхности жидкости с помощью преломленного или отраженного лазерного луча от сравнительно небольшого участка поверхности жидкости \cite{Brazhnikov_IET}. Для регистрации волн на поверхности проводящей жидкости в работе \cite{Falcon2007} в жидкость вводили вертикально ориентированный отрезок изолированной металлической проволоки. В результате образуется цилиндрический конденсатор, одной из обкладок которого служит поверхность проволоки, а другой – проводящая жидкость. По изменению емкости конденсатора со временем можно судить о колебаниях уровня жидкости в точке контакта изолированной проволоки и жидкости.
В работе \cite{Wright1996, Henry2000} камерой регистрировали свет прошедший через полупрозрачную жидкость, а для обеспечения диффузного распространения света в объеме жидкости в рабочую ячейку с жидкостью (водой) вводили полистироловые шарики диаметром 1 мкм или добавляли обычное молоко. На фотографии колеблющейся поверхности яркость отдельных точек определяется высотой уровня поверхности жидкости, т.е. по распределению яркости точек на поверхности можно судить о распределении энергии (амплитуде колебаний) по волновым векторам на поверхности освещаемой снизу "мутной"  жидкости. В работе \cite{Fujimura2008} возбуждение и регистрация колебаний на поверхности воды производится с помощью электрического поля емкостным методом. Для этого на стенке прямоугольной кварцевой кюветы помещены полоски из алюминия, которые играют роль конденсатора.

Остановимся подробнее на методики регистрации волн с помощью отраженного лазерного луча.

\begin{figure}[ht] 
  \center
  \includegraphics [scale=0.15] {Intro/laser.jpg}
  \caption{Схема методики измерения волн на поверхности жидкости с помощью отраженного лазерного луча.} 
  \label{img:laser}
\end{figure}

Колебания поверхности детектируются по схеме, показанной на рисунке \ref{img:laser}. Лазерный луч, отраженный от поверхности жидкости, фокусируется линзой на фотоприемник. Угол скольжения лазерного луча (угол между лазерным лучом и плоскостью поверхности жидкости) к поверхности жидкости  составляет примерно 0.2 рад. Максимальный угол отклонения поверхности жидкости от равновисия составляет 0.05 рад.

В зависимости от характерного размера $a$ пятна лазерного луча на поверхности жидкости и длины волны $\lambda$ детектируемого колебания возможны два метода обработки сигнала с фотоприемника:

1. $ a \ll \lambda$. «Узкий луч». Характерный размер пятна лазерного луча много меньше длины волны. В этом случае мощность отраженного луча зависит от угла отражения, то есть в приближении малых углов мощность принимаемого сигнала фотоприемника линейно зависит от угла отражения.
\begin{equation}
%\label{eq:waveStand}
P(t) \sim R(\alpha + \phi(t)) \approx R(\alpha) + const \phi(t)
\end{equation}
Для квадрата Фурье-компонент:
\begin{equation}
%\label{eq:waveStand}
P^2_\omega \sim \phi^2_\omega
\end{equation}
где $P(t)$ - мощность сигнала на фотоприемнике, $R$ – коэффициент отражения.

2. $a \gg \lambda$. «Широкий луч». Характерный размер пятна лазерного луча много больше длины волны. В этом случае мощность отраженного луча является интегральной характеристикой поверхности. В результате усреднения получится следующая зависимость \cite{Brazhnikov_IET}:
\begin{equation}
%\label{eq:waveStand}
P^2_\omega \sim I_\omega
\end{equation}

\section{Возбуждение вихревых течений поверхностными волнами}% \label{sect3_1}

Сравнительно недавно было обнаружено, что при параметрическом возбуждении  наряду с волновым движением на поверхности жидкости также наблюдается течение, демонстрирующее хаотическое поведение \cite{Ramshankar1990}. Впоследствии было показано, что это течение соленоидально и с увеличением амплитуды волн может оказаться достаточно интенсивным для формирования турбулентного каскада \cite{VonKameke2011, Francois2014, Francois2013} подобно обратному каскаду в двумерной турбулентности \cite{Kraichnan1967}. Несмотря на большое количество экспериментальных исследований, посвященных волнам Фарадея, природа возникновения в них течения до настоящего времени не была выяснена. В работе \cite{Mesquita1992} это течение пытались описать как средний дрейф Стокса \cite{Stokes1847} для случайного волнового поля. Однако найденное в эксперименте значение коэффициента диффузии пассивного скаляра почти на порядок превышало теоретическое.	
	Наше исследование показало \cite{F5}, что генерация вихревого движения не является особенностью неустойчивости Фарадея, а является следствием нелинейного взаимодействия волн, распространяющихся под углом друг к другу. Таким образом данное явление уже имеет отношение к движению поверхности океана. В частности оно может играть значительную роль в перемешивании планктона и в движении загрязняющих веществ на поверхности воды \cite{Falkovich2009}.
	Стоит еще отметить, что изучив механизм генерации вихревых течений поверхностными волнами можно научиться создавать вихревое движение заданное формы, возбуждая на поверхности волны силой с  рассчитанной спектральной характеристикой.
	
Для количественного изучения вихревых движений используется величина завихренности, определяемая как:

\begin{equation}
 \label{eq:defVort}
\Omega(x, y) = \frac{\partial V_x}{\partial y} - \frac{\partial V_y}{\partial x}
\end{equation}
где $V_x$, $V_y$ – компоненты скорости жидкости. 






%Кроме того, как выяснилось, при сильном возбуждении наряду с нелинейным взаимодействием волн наблюдается генерация вихревого движения \cite{Shats2005, VonKameke2011}.
% Недавно в [F5, F6] было показано, что завихренность формируется в результате взаимодействия нелинейных волн, имеющих непараллельные волновые векторы, т.е. в двумерном пространстве волновых векторов k. В \cite{Brazhnikov_liq_hydr} волны на поверхности цилиндрической ячейки возбуждали с помощью кольца, касающегося поверхности воды вблизи стенок ячейки. На поверхности возбуждалась только радиальная мода. В этом случае стоячие волны на поверхности описываются функцией Бесселя параметра $Rk$ ($R$ – радиус ячейки). Скаляр $k$ играет роль волнового числа: при больших значениях $R/\lambda$, ($\lambda = 2*\pi/k$ – длина возбуждаемой волны) и на большом расстоянии $r \gg \lambda$ от центра ячейки в узком угловом секторе цилиндрическую волну можно рассматривать как плоскую волну с волновым числом k в одномерном k-пространстве. Экспериментальные результаты \cite{Brazhnikov_liq_hydr} оказались в хорошем согласии с теорией слабой (волновой) турбулентности \cite{Zakharov}.

%В настоящем сообщении представлены экспериментальные результаты исследований турбулентности в системе волн на поверхности воды, возбуждаемых вертикальными колебаниями ячейки за счет краевого эффекта смачивания при ускорениях меньше порового значения возникновения неустойчивости Фарадея в цилиндрической ячейке, когда вихревое движение еще не наблюдается, и в квадратной ячейке, где вихревое движение при этих уровнях накачки хорошо развито.
%Гидродинамика жидкости со свободной поверхностью давно является предметом теоретических и экспериментальных исследований. 
\section{Дрейф Стокса} \label{p1_Stockes}

Волновое движение на свободной поверхности жидкости описывается гармоническим колебанием, затухающим в глубины \cite{land}:

\begin{equation}
 \label{eq:waveSimple}
 V_x(x,z,t) = -A k e^{kz}sin(kx-\omega t)
\end{equation}

\begin{equation}
 V_z(x,z,t) = A k e^{kz}cos(kx-\omega t)
\end{equation}

Несмотря на то, что среднее по периоду колебания значение скорости в любой точке пространства равно 0, пробная частичка, помещенная в жидкость будет иметь не нулевое смещение за период колебания. Это связано с тем, что поле скорости жидкости меняется со временем, и для подсчета величины смещения пробной частички  в пространстве необходимо интегрировать скорость жидкости по траектории движения пробной частички. В результате изменения скорости жидкости со временем траектория пробной частички получается незамкнутой и смещающейся за каждый период, см. рис. \ref{img:stockesTrack}.

\begin{figure}[ht] 
  \center
  \includegraphics [scale=0.2] {Intro/StockesTrack.jpg}
  \caption{} 
  \label{img:stockesTrack}  
\end{figure}

Другими словами пробная частичка в волне будет совершать колебания, смещаясь в сторону распростронения волны. Средную скорость смещения можно оценить как \cite{FalkovichBook}:
\begin{equation}
 \label{eq:StockesVel}
	V_{St} =  A^2 k^3 /2 \omega
\end{equation}

Бегующая волна создает перенос массы с направлении своего распростронения, который прекращается тут же как прекращается волновое движение в данной точке пространства.

Представление уравнений движения жидкости, когда в качестве скоростей жидкости испульзуется скорости жидкости в опредленных точках пространства - называется эйлеровым. В лагранжевом предствалении исследуется скорости определенных кусочков жидкости. При движении бегущей волны средняя эйлерова скорости будет равна нулю, в то время как средняя лагранжева скорость будет равна дрейфу Стокса.

При регистрации движения декорирующих частиц на поверхности жидкости может регистрироваться как лагранжева скорость, так и эйлерова. Для регистрации эйлеровой скорости необходимо выполнение следующих условий: частота съемки должна быть много больше частоты волны и шаг пространственной сетки регистрации должен быть много меньше амплитуды колебания кусочка жидкости за период волны. Так как второе условие в наших экспериментах технически выполнить сложно, то стоит считать, что в наших экспериментах измеряется лагранжева скорость.










%
%
%
%
%
%
%\clearpage