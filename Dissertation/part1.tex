\chapter{Введение} \label{chapt1}

\section{Волновая турбулентность} \label{sect1_1}
\subsection{Закон дисперсии волн на поверхности жидкости} \label{subsect1_1_1}

Волны на поверхности жидкости формируются за счет силы гравитации и сил поверхностного натяжения, причем влияние гравитации преобладает при больших длинах волн, а капиллярные силы при малых, что видно из закона дисперсии для поверхностных волн: 
\begin{equation}
 \label{eq:disper}
\omega^2 = (gk + \sigma/\rho k^3)th(kh),
\end{equation}
где h - глубина жидкости.

В случае, когда $k \gg (g\rho/\sigma)^{1/2}$ влияние гравитационных сил становится пренебрежимо малым по сравнению 
с капиллярными силами. Такие волны называют капиллярными. Если $k \ll (g\rho/\sigma)^{1/2}$, то волны называются 
гравитационными. В промежуточном случае говорят о капиллярно-гравитационных волнах. Для свободной поверхности воды 
характерная частота перехода от гравитационных волн к капиллярным соответствущая волновому вектору 
$(g\rho/\sigma)^{1/2}$ будет ~ 17 Гц, длина волны ~ 1.5 см. Для поверхности жидкого водорода - ???.

Стоит отметить, что так как в проводимых  нами экспериментах с капиллярными волнами характерная глубина жидкости была около 1 см, а характерные волновые вектора были больше 10 см$^{-1}$, то $th(kh)$ можно считать равным 1.
Таким образом, учитывая, что наши эксперименты проводились в области высоких частот, закон дисперсии можно считать капиллярным для глубокой воды:
\begin{equation}
 \label{eq:disperCap}
\omega^2 = \sigma/\rho k^3
\end{equation}

В экспериментах с гравитационными волнами глубина жидкости была около $ h \sim 7$ см, минимальный волной вектор $k = 0.36$, соответсвенно $th(kh) \sim 0.99$, т.е. влиянием глубины можно также пренебречь. Таким образом закон дисперсии будет:

\begin{equation}
 \label{eq:disperGrav}
\omega^2 = (gk + \sigma/\rho k^3),
\end{equation}



\subsection{Законы сохранения} \label{subsect1_1_2}

При взаимодействии волн должны выполняться законы сохранения энергии и импульса. Соответственно для капиллярных волн трехволновые процессы оказываются разрешенными, т.е. закон дисперсии является распадным. Законы сохранения энергии и импульса для трехволнового процесса можно записать как:
\begin{equation}
 \label{eq:saveOmega}
\omega_1 \pm \omega_2 = \omega_3
\end{equation}
\begin{equation}
 \label{eq:saveK}
\mbox{\boldmath$k_1$} \pm \mbox{\boldmath$k_2$} = \mbox{\boldmath$k_3$}
\end{equation}

Таким образом при возбуждении на поверхности жидкости волн в области низких частот может быть сформировано турбулентное состояние в котором поток энергии направлен от области низких частот(область накачки) в сторону больших частот. Теория слабой волновой турбулентности [zakharov?] предсказывает, что основной вклад в перенос энергии по турбулентному капиллярному каскаду вносят трехволновые процессы слияния волн. 

\subsection{Вязкость в волновой системе} \label{subsect1_1_3}
Вязкость для волн с разными частотами растет с увеличением частоты, таким образом энергия в турбулентном каскаде поверхностных волн передается в сторону больших частот до тех пор, пока вязкие потери не становятся сравнимы с потоком энергии.

Таким образом в турбулентной системе можно выделить три характерные области: область накачки, в которой энергия приходит в систему, инерционный интервал, где энергия передается практически без потерь и область диссипации, где энергия покидает систему. 

В настоящий момент имеется довольно много теоретических и экспериментальных работ посвященных изучению ИИ турбулентного каскада в различных системах<...>. Теория волновой турбулентности предсказывает степенное распределение энергии по шкале частот:
\begin{equation}
% \label{eq:EOmega}
E_\omega \sim \omega^{-\alpha}
\end{equation}

С экспериментальной точки зрения удобно исследовать не распределение энергии по волновым векторам(или частотам) напрямую, парную корреляционную функцию отклонения поверхности от положения равновесия I()=<(r, t+)(r,t)>, так как величину отклонения поверхности от положения равновесия можно непосредственно экспериментально измерить. Фурье образ парной корреляционную функцию отклонения поверхности от равновесного состояния связан с распределением энергии по частотам:
\begin{equation}
% \label{eq:EOmega}
E_\omega \sim \omega^{-4/3}I_\omega
\end{equation}

Таким образом предсказывается степенная зависимость $I_\omega \sim \omega^{-m}$

В зависимости от характера накачки теория волновой турбулентности предсказывает различные показатели m. Для широкополосной накачки (когда ширина полосы накачки сопоставима или больше самой частоты накачки), предсказывается m = 17/6. При накачке узкополосным сигналом в спектре появляются равноудаленные пики, максимумы которых убывают c ростом частоты с показателем m = 23/6. Данные предсказания подтверждаются с помощью компьютерного моделирования, так в экспериментальных исследованиях. Форма инерционного интервала в спетрах турбулентных каскадов в системе волн на поверхности воды, жидкого водорода и жидкого гелия экспериментально хорошо изучены в работах <...>. 

Стоит отметить, что одной из сложностью для экспериментального исследования и проведения вычислетельных экспериментов по исследования турбулетнтых каскадов явлется степенное уменьшения энергии волны с ростом частоты. Так как величина показателя степени m находится в районе 2-3, а диапазон частот в котором существует турбулентный каскад может достигать нескольких декад. то для экспериментального исследования поведения турбулентной системы в достаточно широком диапазоне частот, необходимым для наблюдения развитого турбулентного каскада, требуется экспериментальное оборудование обладающее большим динамическим диапазоном. Таким образом экспериментальное исследование ИИ и диссипативной области турбулентных каскадов было практически невозможно до появления широко распространенных АЦП с высоким динамическим диапазоном и достаточной высокой частотой обработки, а также развитием компьютерной техники для обработки полученных сигналов.
\subsection{Диссипативная область турбулентного каскада} \label{subsect1_1_4}

Для определения частотной области в которой будет происходить основная диссипация энергии рассмотрим такие важные для изучении турбулентных каскадов характеристики как вязкое время и нелинейное время.

Вязкое время определяется как характерное время вязкого затухания волны с заданным волновым вектором.
\begin{equation}
% \label{eq:tauNu}
1/\tau_\nu = 2\nu k^2 = 2 \nu \omega^{4/3}(\sigma/\rho)^{2/3}
\end{equation}
Характерное время нелинейного взаимодействия капиллярных волн можно выразить через параметры жидкости и функцию распределения капиллярных волн $n(\omega)$:
\begin{equation}
% \label{eq:tauNu}
1/\tau_nl = |V_\omega|^2 n(\omega)
\end{equation}

Где $V(\omega) \approx (\sigma/\rho)^{3/2}\omega^{3/2}$ - коэффициент трехволнового нелинейности капиллярных волн

Функцию распределения капиллярных волн $n(\omega) \sim E_{omega}\omega^{-1}$ 

Положение высокочастотной границы ИИ можно оценить из предположения совпадения по порядку величины времени нелинейного взаимодействия волн и времени вязкого затухания на частоте границы.

Получаем оценку граничной частоты ИИ $\omega_b \sim \eta_p^{2.4}$ для широкополосной накачки и bp4/3для узкополосной накачки. Несмотря на то, что отдельные экспериментальные работы по измерению поведения положения границы ИИ производились <>, работ направленных на создание общей картины поведения положения границы ИИ для разных типов накачки и разной геометрии экспериментальной ячейки не было.
%	Отличие водорода от воды. Почему водород и почему вода?

Диссипативная область турбулентного каскада.
%Несмотря на то, что в диссипативной области вязкое время превышает нелинейное и преобладают процессы затухания нелинейные процессы осуществляют существенное влияние на поведение спектра. Если волны в диссипативном интревале взаимодействуют в основном с ближайшими соседями, а не волнами из инерционного интервала, то распределение энергии по волнам в области диссипации становиться близким к экспоненциальному. В <...> был проведен детальный анализ, который дал квазипланковский спектр корреляционной функций в диссипативной области:
%Psexp(-d), где d- характерная частота распределения.

\section{Дрейф Стокса} \label{sect1_3}


\clearpage