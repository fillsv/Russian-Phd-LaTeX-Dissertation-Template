\chapter{Введение} \label{chapt1}
геометрия волн в зависимости от геомертрии ячейки
\section{Волновая турбулентность} \label{sect1_1}
\subsection{Закон дисперсии волн на поверхности жидкости} \label{subsect1_1_1}

Волны на поверхности жидкости формируются за счет силы гравитации и сил поверхностного натяжения, причем влияние гравитации преобладает при больших длинах волн, а капиллярные силы при малых, что видно из закона дисперсии для поверхностных волн: 
\begin{equation}
 \label{eq:disper}
\omega^2 = (gk + \sigma/\rho k^3)tanh(kh),
\end{equation}
где h - глубина жидкости.

В случае, когда $k \gg (g\rho/\sigma)^{1/2}$ влияние гравитационных сил становится пренебрежимо малым по сравнению 
с капиллярными силами. Такие волны называют капиллярными. Если $k \ll (g\rho/\sigma)^{1/2}$, то волны называются 
гравитационными. В промежуточном случае говорят о капиллярно-гравитационных волнах. Для свободной поверхности воды 
характерная частота перехода от гравитационных волн к капиллярным соответствущая волновому вектору 
$(g\rho/\sigma)^{1/2}$ будет ~ 17 Гц, длина волны ~ 1.5 см. Для поверхности жидкого водорода - ???.

Стоит отметить, что так как в проводимых  нами экспериментах с капиллярными волнами характерная глубина жидкости была около 1 см, а характерные волновые вектора были больше 10 см$^{-1}$, то $th(kh)$ можно считать равным 1.
Таким образом, учитывая, что наши эксперименты проводились в области высоких частот, закон дисперсии можно считать капиллярным для глубокой воды:
\begin{equation}
 \label{eq:disperCap}
\omega^2 = \sigma/\rho k^3
\end{equation}

В экспериментах с гравитационными волнами глубина жидкости была около $ h \sim 7$ см, минимальный волной вектор $k = 0.36$, соответсвенно $th(kh) \sim 0.99$, т.е. влиянием глубины можно также пренебречь. Таким образом закон дисперсии будет:

\begin{equation}
 \label{eq:disperGrav}
\omega^2 = (gk + \sigma/\rho k^3),
\end{equation}



\subsection{Законы сохранения} \label{subsect1_1_2}

При взаимодействии волн должны выполняться законы сохранения энергии и импульса. Соответственно для капиллярных волн трехволновые процессы оказываются разрешенными, т.е. закон дисперсии является распадным. Законы сохранения энергии и импульса для трехволнового процесса можно записать как:
\begin{equation}
 \label{eq:saveOmega}
\omega_1 \pm \omega_2 = \omega_3
\end{equation}
\begin{equation}
 \label{eq:saveK}
\mbox{\boldmath$k_1$} \pm \mbox{\boldmath$k_2$} = \mbox{\boldmath$k_3$}
\end{equation}

Таким образом при возбуждении на поверхности жидкости волн в области низких частот может быть сформировано турбулентное состояние в котором поток энергии направлен от области низких частот(область накачки) в сторону больших частот. Теория слабой волновой турбулентности [zakharov?] предсказывает, что основной вклад в перенос энергии по турбулентному капиллярному каскаду вносят трехволновые процессы слияния волн. 

\subsection{Вязкость в волновой системе} \label{subsect1_1_3}
Вязкость для волн с разными частотами растет с увеличением частоты, таким образом энергия в турбулентном каскаде поверхностных волн передается в сторону больших частот до тех пор, пока вязкие потери не становятся сравнимы с потоком энергии.

Таким образом в турбулентной системе можно выделить три характерные области: область накачки, в которой энергия приходит в систему, инерционный интервал, где энергия передается практически без потерь и область диссипации, где энергия покидает систему. 

В настоящий момент имеется довольно много теоретических и экспериментальных работ посвященных изучению ИИ турбулентного каскада в различных системах<...>. Теория волновой турбулентности предсказывает степенное распределение энергии по шкале частот:
\begin{equation}
% \label{eq:EOmega}
E_\omega \sim \omega^{-\alpha}
\end{equation}

С экспериментальной точки зрения удобно исследовать не распределение энергии по волновым векторам(или частотам) напрямую, парную корреляционную функцию отклонения поверхности от положения равновесия I()=<(r, t+)(r,t)>, так как величину отклонения поверхности от положения равновесия можно непосредственно экспериментально измерить. Фурье образ парной корреляционную функцию отклонения поверхности от равновесного состояния связан с распределением энергии по частотам:
\begin{equation}
% \label{eq:EOmega}
E_\omega \sim \omega^{-4/3}I_\omega
\end{equation}

Таким образом предсказывается степенная зависимость $I_\omega \sim \omega^{-m}$

В зависимости от характера накачки теория волновой турбулентности предсказывает различные показатели m. Для широкополосной накачки (когда ширина полосы накачки сопоставима или больше самой частоты накачки), предсказывается m = 17/6. При накачке узкополосным сигналом в спектре появляются равноудаленные пики, максимумы которых убывают c ростом частоты с показателем m = 23/6. Данные предсказания подтверждаются с помощью компьютерного моделирования, так в экспериментальных исследованиях. Форма инерционного интервала в спетрах турбулентных каскадов в системе волн на поверхности воды, жидкого водорода и жидкого гелия экспериментально хорошо изучены в работах <...>. 

Стоит отметить, что одной из сложностью для экспериментального исследования и проведения вычислетельных экспериментов по исследования турбулетнтых каскадов явлется степенное уменьшения энергии волны с ростом частоты. Так как величина показателя степени m находится в районе 2-3, а диапазон частот в котором существует турбулентный каскад может достигать нескольких декад. то для экспериментального исследования поведения турбулентной системы в достаточно широком диапазоне частот, необходимым для наблюдения развитого турбулентного каскада, требуется экспериментальное оборудование обладающее большим динамическим диапазоном. Таким образом экспериментальное исследование ИИ и диссипативной области турбулентных каскадов было практически невозможно до появления широко распространенных АЦП с высоким динамическим диапазоном и достаточной высокой частотой обработки, а также развитием компьютерной техники для обработки полученных сигналов.
\subsection{Диссипативная область турбулентного каскада} \label{subsect1_1_4}

Для определения частотной области в которой будет происходить основная диссипация энергии рассмотрим такие важные для изучении турбулентных каскадов характеристики как вязкое время и нелинейное время.

Вязкое время определяется как характерное время вязкого затухания волны с заданным волновым вектором.
\begin{equation}
% \label{eq:tauNu}
1/\tau_\nu = 2\nu k^2 = 2 \nu \omega^{4/3}(\sigma/\rho)^{2/3}
\end{equation}
Характерное время нелинейного взаимодействия капиллярных волн можно выразить через параметры жидкости и функцию распределения капиллярных волн $n(\omega)$:
\begin{equation}
% \label{eq:tauNu}
1/\tau_nl = |V_\omega|^2 n(\omega)
\end{equation}

Где $V(\omega) \approx (\sigma/\rho)^{3/2}\omega^{3/2}$ - коэффициент трехволнового нелинейности капиллярных волн

Функцию распределения капиллярных волн $n(\omega) \sim E_{omega}\omega^{-1}$ 

Положение высокочастотной границы ИИ можно оценить из предположения совпадения по порядку величины времени нелинейного взаимодействия волн и времени вязкого затухания на частоте границы.

Получаем оценку граничной частоты ИИ $\omega_b \sim \eta_p^{2.4}$ для широкополосной накачки и bp4/3для узкополосной накачки. Несмотря на то, что отдельные экспериментальные работы по измерению поведения положения границы ИИ производились <>, работ направленных на создание общей картины поведения положения границы ИИ для разных типов накачки и разной геометрии экспериментальной ячейки не было.
%	Отличие водорода от воды. Почему водород и почему вода?

Диссипативная область турбулентного каскада.
Несмотря на то, что в диссипативной области вязкое время превышает нелинейное и преобладают процессы затухания нелинейные процессы осуществляют существенное влияние на поведение спектра. Если волны в диссипативном интревале взаимодействуют в основном с ближайшими соседями, а не волнами из инерционного интервала, то распределение энергии по волнам в области диссипации становиться близким к экспоненциальному. В <...> был проведен детальный анализ, который дал квазипланковский спектр корреляционной функций в диссипативной области:
\begin{equation}
% \label{eq:tauNu}
P_\omega \sim exp(-\omega/\omega_d),
\end{equation}
где $\omega_d$ - характерная частота распределения.
\section{Введение} %\label{sect1_1}
Tеория слабой турбулентности описывает многочисленные системы слабовзаимодействующих волн: рябь на воде и гравитационные волны на поверхности океана, волны Россби в атмосфере планет и в мировом океане, Ленгмюровские волны в плазме и спиновые волны в магнетиках. Сравнительно низкая плотность жидкого водорода, его низкая низкая плотность и возможность возбуждать волны на заряженной поверхности электрическим полем приводят к уникальной возможности экспериментального изучения слабой волновой турбулентности. Использование жидкого водорода в экспериментах по волновой турбулентности уже помогло нам исследовать явления предсказанные теорией, например стационарный  спектр Захакрова-Колмогорова в капиллярной турбулентности в широком диапазоне частот, а также наблюдать новые, которые были успешно объяснены в рамках  приближения слабой турбулентности: квазиадиабатический распад капиллярной турбулентности и подавление высокочастотных турбулентных колебаний добавлением низкочастотной возбуждающей силы.
Слабая волновая теория предсказывает существование стационарного неравновесного состояния в системе взаимодействующих капиллярных волн - спектр Колмогорова-Захарова для спектральной плотности энергии $\varepsilon(k)$ волн в инерционном интервале.
$\varepsilon(k) \sim P^{1/2}k^{-7/4}$.

Используя соотношение между спектральной плотностью энергии $\varepsilon(k)$ и парной корреляционной функцией высоты поверхности $\eta(\mathbf{r}, t)$, и принимая во внимание закон дисперсии капиллярных волн $omega(k) \sim k^{3/2}$, можно получить частотный спектр корреляционной функции $<\eta(t+\tau)\eta(t)>$:
\begin{equation}
% \label{eq:tauNu}
<\eta_\omega^2> \sim (\sigma k^2)^{-1} \varepsilon(k)(d\omega/dk)^{-1} \sim P^{1/2} \omega^{-17/6}.
\end{equation}

 

Этот стационарный спектр характеризуется одной величиной $P$ - потоком энергии на большие частоты $\omega$. На больших частотах передаваемая энергия диссипирует из-за вязких потерь и турбулентный каскад разрушается. Следовательно для поддержания распределения (2) или (1) постоянным во времени, система должна постоянно накачиваться энергией на низких частотах. Высокочастотная граница инерционного интервала $\omega_d$, где распределение (2) еще остается верным, может быть оценена из предположения, что время вязкого затухания $\tau_\nu(\omega)$ и время нелинейного взаимодействия $\tau_{nl}(\omega)$ равны по порядку для волн на частоте $\omega_d$. Это предположение дает нам [5]:
\begin{equation}
% \label{eq:tauNu}
\omega_d	 \sim (\frac{P^{1/2}}{\nu})^{6/5} \sim (\eta_0^2\omega_0^{17/6}/\nu)^{6/5},
\end{equation}

Где, $\nu_0^2$- среднеквадратичная амплитуда волны на некоторой низкой частоте $\omega_0$, $\nu$ - кинематическая вязкость жидкости.

На частотах больших $\omega_d$ спектр зависит и от специфики затухания и от нелинейного взаимодействия. Когда волна в диссипативной области $\omega \gg \omega_d$ взаимодействует в основном с другими волнами диссипативной области, нежели волнами из инерционного интервала $\omega \ll \omega_d$ , волновое распределение будет похоже на экспоненциальное. Детальное рассмотрение дает “квазипланковский” спектр для диссипативной области корреляционной функции [7]:
\begin{equation}
% \label{eq:tauNu}
<\eta_\omega^2> \sim \omega^-s e^{-\omega/\omega_d},
\end{equation}				(4)
где s - некая константа. Численные вычисления для дискретного кинетического уравнения [7] подтверждают экспоненциальную зависимость волнового числа заполнения в области сильного затухания.
	
	Мы представляем первое экспериментальное наблюдение турбулентных спектров капиллярных волн в диссипативной области возбужденных низкочастотной случайной силой на поверхности жидкого водорода.

\section{Экспериментальная методика} %\label{sect1_1}
 Экспериментальная установка состоит из оптической ячейки расположенной в вакуумной полости гелиевого криостата и оптической системы регистрации колебаний на поверхности жидкости. Проводящий цилиндрический сосуд 6 мм глубиной и 60 мм диаметром был установлен внутри ячейки, а проводящая пластина зафиксирована в 4 мм над сосудом. Газообразный водород сконденсирован в сосуде. Радиоактивная пластина, расположенная в нижней части сосуда, ионизирует жидкий водород. В присутствии постоянного напряжения около 1 кВ между сосудом и верхней пластиной, ионы разделяются и заряды со знаком соответствующим полярности постоянного напряжения собираются под поверхностью жидкого водорода. Волны на поверхности жидкого водорода возбуждаются добавочное переменное напряжение с максимальной амплитудой 100 В.

	Использование электрического поля для создания капиллярных волн дает большое преимущество по сравнению с использованием техник возбуждения волн на поверхности воды, например неустойчивость Фарадея. Это позволяет нам возбуждать поверхность аккуратно достаточно хорошо контролируемой силой. В этой серии экспериментов в качестве переменного возбуждающего напряжения были использованы низкочастотные случайные сигналы. Эти сигналы были синтезированы через фурье-преобразование случайного набора фаз и прямоугольного амплитудного спектра, которые везде равен нулю кроме заданного частотного интервала(интервал накачки).

	Для регистрации волн на поверхности жидкости использовался метод отражения лазерного луча. Лазерный луч падает под малым скользящим лучом(около 0.2 рад) на поверхность жидкости, отражается и фокусируется линзой на фотодетектор. Напряжение на фотодетекторе усиливается и оцифровывается 24 битным аналого-цифровым преобразователем с частотой дискретизации около 100 кГц. Волны регистрировались в режиме "широкого луча", когда размер лазерного луча больше, чем характерная длина волны. Энергия отраженного лазерного луча $P(t)$  в этом режиме пропорциональна отклонению поверхности $\eta(t)$ [8]. По этой причине в дальнейшем не делается разницы между спектром корреляционной функции отклонения поверхности $<|\eta_\omega^2|>$ и энергии отраженного лазерного луча $<P_\omega^2>$ .

	Максимальная крутизна волны которая может быть зарегистрирована в эксперименте ограничена размером оптических окон криостата и приблизительно равно 0.05 рад.

\section{Экспериментальные результаты и обсуждение} %\label{sect1_1}
 Капиллярные волны возбуждались случайной силой в частотном диапазоне 39-103 Гц. Средний квадрат возбуждающего напряжения менялся от $V_p = 0$ В, т.е. отсутствие накачки, до $V_p = 30$ В , ограничение связано с максимальной крутизной волны. На рис показан пример фурье спектра для отраженной энергии лазерного луча $P_\omega^2$ для разных амплитуд возбуждающей силы. На рис хорошо видно область накачки на низкочастотной части спектра. За областью накачки следует инерционный интервал - относительно широкая частотная область, где видна степенная зависимость спектра $P_\omega^2$. Ширина инерционного интервала зависит от амплитуды накачки. Когда поверхность возбуждается слабо (переменное напряжение $V_p = 4$ В) диссипация начинается рядом с областью накачки и инерционный интервал не наблюдается. Увеличение силы накачки приводит к уширению инерционного интервала, высокочастотная граница инерционного интервала $\omega_b$ смещается к высоким частотам. Наиболее широкий инерционный интервал с границами от $\approx 0.3$ кГц, до $\omega_b \approx 4$ кГц наблюдается при максимальном напряжении накачки $V_p = 30$ В. На частотах выше высокочастотной границы колебания поверхности затухают из-за вязких потерь, кривая $P_\omega^2$ идет вниз гладко и уходит ниже уровень аппаратных шумов.

	Турбулентные спектры перестроенные в линейном масштабе на рис показывают, что убывание амплитуд волн с частотой после высокочастотной границы инерционного интервала может быть достаточно хорошо описано экспоненциальным затуханием $P_\omega^2 \sim	e^{-\omega/\omega_d}$ в некотором интервале. Подгонка согласуется с начальным предположения, что $\omega \gg \omega_d$, полученный параметр $\omega_d$ значительно меньше, чем частоты из интервала подгонки. Например спектр при напряжении накачки $V_p = 26$ В подгонялся в диапазоне 5-9 кГц с $\omega_d \approx 0.6$ кГц. К сожалению узкий интервал подгонки не позволяет установить показатель степени s “квазипланковского” распределения достаточно точно. Полученные значения $\omega_d$ в несколько раз меньше, чем видимая граница между инерционным интервалом и диссипативной области(см рис). Это несоответствие можно отнести к определенной степени свободы в определении граничной частоты, которая может быть перенормирована с помощью некоторой константы.

	Граница вязкого затухания $\omega_d$ наблюденная с помощью подгонки экспоненциального затухания в диссипативной области растет с увеличением возбуждающей силы. Для измерения уровня возбуждения использовался отклик поверхности $\eta_0$, а именно абсолютное значение $P_\omega$ на частоте 53 Гц (положение максимума распределения $P_\omega^2$ внутри области накачки). Величина $\eta_0$ прямопрорциональна средней высоте волны на той же самой частоте. На рис показано, что зависимость граничной частоты от величины возбуждения может быть описана степенным законом $\omega_d(\eta_0) \sim	\eta_0^m$ со значение показателя $m = 0.85 \pm 0.05$. Необходимо заметить, что подгонка экспоненциальных спектров с помощью “квази-Планка” с малым ненулевым $s$ ($|s| \le 2$) слабо влияет на полученный параметр $\omega_d$ (меньше чем на 20\%). Однако эта поправка не изменит показатель степени $m$ в пределах погрешности.

	Наблюдаемый показатель $m \approx 0.85$ значительно отличается от ожидаемого $m = 12/5$ из формулы (3). Стоит отметить, что в случае турбулентных каскадов, возбужденных монохроматической силой, измеренная граничная частота находится в хорошем соответствии с ожиданиями $\omega_d(\eta) \sim \eta^{1.3}$.

\section{Выводы} %\label{sect1_1}
 	Заключение. Впервые наблюден переход от степенного в инерционном интервале спектра Колмогова-Захарова к “квазипланковскому” распределению $\omega^{-s}e^{-\omega/\omega_d}$ в области диссипации для капиллярной турбулентности. Экспоненциальный спад в области диссипации $\omega/\omega_d \gg 1$ соответствует теоретическому ожиданию и качественно соответствует численным вычислениям [7]. Граница вязкого затухания $\omega_d$ растет с увеличением амплитуды накачки и зависит от средней высоты волны $\eta_0$ на частоте накачки как $\omega_d \sim \eta^{0.85 \pm 0.05}$	. Однако наблюденная зависимость отличается от ожидаемой, показатель степени почти в три раза больше, чем предсказанное значение.


\clearpage