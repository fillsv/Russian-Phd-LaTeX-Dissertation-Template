\chapter{Введение}\label{chapt1}
\section{Волновая турбулентность}% \label{subsect1_turb}
Теория слабой волновой турбулентности описывает многочисленные системы слабовзаимодействующих волн: рябь на воде и гравитационные волны на поверхности океана, волны Россби в атмосфере планет и в мировом океане, Ленгмюровские волны в плазме и спиновые волны в магнетиках.

<Прямой каскад, модель Колмогорова>
\section{Закон дисперсии волн на поверхности жидкости}% \label{subsect1_disper}

Волны на поверхности жидкости формируются за счет силы гравитации и сил поверхностного натяжения, причем влияние гравитации преобладает при больших длинах волн, а капиллярные силы при малых, что видно из закона дисперсии для поверхностных волн: 
\begin{equation}
 \label{eq:disper_dip}
\omega^2 = (gk + \sigma/\rho k^3)th(kh),
\end{equation}
где h - глубина жидкости.

В случае, когда $k \gg (g\rho/\sigma)^{1/2}$ влияние гравитационных сил становится пренебрежимо малым по сравнению 
с капиллярными силами. Такие волны называют капиллярными. Если $k \ll (g\rho/\sigma)^{1/2}$, то волны называются гравитационными. В промежуточном случае говорят о капиллярно-гравитационных волнах. Для свободной поверхности воды характерная частота перехода от гравитационных волн к капиллярным соответствует волновому вектору $(g\rho/\sigma)^{1/2}$ и составляет $\sim$ 17 Гц, а длина волны $\sim$ 1.5 см. Для поверхности жидкого водорода граничная частота также равна $\sim$ 17 Гц, что соответствует длине волны $\sim$ 1.1 см.


%Сравнительно низкая плотность жидкого водорода и возможность возбуждать волны на заряженной поверхности электрическим полем приводят к уникальной возможности экспериментального изучения слабой волновой турбулентности. Использование жидкого водорода в экспериментах по волновой турбулентности уже помогло исследовать явления предсказанные теорией, например стационарный  спектр Захакрова-Колмогорова в капиллярной турбулентности в широком диапазоне частот[], а также наблюдать новые, которые были успешно объяснены в рамках  приближения слабой турбулентности: квазиадиабатический распад капиллярной турбулентности[] и подавление высокочастотных турбулентных колебаний добавлением низкочастотной возбуждающей силы[].
Сравнительно низкая плотность жидкого водорода, его низкая низкая плотность и возможность возбуждать волны на заряженной поверхности электрическим полем приводят к уникальной возможности экспериментального изучения слабой волновой турбулентности. Использование жидкого водорода в экспериментах по волновой турбулентности уже помогло исследовать явления предсказанные теорией, например стационарный  спектр Захакрова-Колмогорова в капиллярной турбулентности в широком диапазоне частот, а также наблюдать новые, которые были успешно объяснены в рамках  приближения слабой турбулентности: квазиадиабатический распад капиллярной турбулентности \cite{quasiadiabatic} и подавление высокочастотных турбулентных колебаний добавлением низкочастотной возбуждающей силы \cite{addLowFreq}.

Стоит отметить, что если глубина жидкости больше, чем характерная длина волны, то $kh > 2\pi$ и $th(kh)$ можно считать равным 1. Таким образом, в приближении глубокой воды дисперсия гравитационно-капиллярных волн записывается как:


\begin{equation}
 \label{eq:disper}
\omega^2 = gk + \sigma/\rho k^3,
\end{equation}


В экспериментах с гравитационными волнами глубина жидкости была около $ h \sim 7$ см, минимальный волной вектор $k = 0.36$ см$^{-1}$, соответственно $th(kh) \sim 0.99$, т.е. влиянием глубины также можно пренебречь.


В области высоких частот, закон дисперсии будет капиллярным:
\begin{equation}
 \label{eq:disperCap}
\omega^2 = \sigma/\rho k^3
\end{equation}

%В экспериментах с гравитационными волнами глубина жидкости была около $ h \sim 7$ см, минимальный волной вектор $k = 0.36$, соответсвенно $th(kh) \sim 0.99$, т.е. влиянием глубины можно также пренебречь. Таким образом закон дисперсии будет:
%\begin{equation}
% \label{eq:disperGrav}
%\omega^2 = gk,
%\end{equation}

\section{Разрешенные моды в ограниченной геометрии} %\label{subsect1_geometr}
%геометрия волн в зависимости от геомертрии ячейки
Рисунок волн на поверхности сильно зависит от геометрии сосуда и способа возбуждения волн.
В случае, если длина затухания волны много больше характерного размера ячейки, при возбуждении волн на поверхности жидкости будет система стоячих волн. Форма стоячих волн будет зависеть от граничных условий и возбуждаемой моды. Граничным условием для волн в ограниченной ячейке является неспособность воды проходить через стенку ячейки, т.е. нормальная (к стенке ячейки) компонента скорости жидкости равна нулю. Так для цилиндрической ячейки резонансные моды волн будут описываться функцией Бесселя:

\begin{equation}
 \label{eq:Bessel}
h(r, \phi, t) = A J(kr) cos(\phi n) cos(\omega t)
\end{equation}

Если n = 0, то мода будет радиальной, в таком случае скаляр $k$ играет роль волнового числа: при больших значениях $R/\lambda$, ($\lambda = 2\pi/k$ – длина возбуждаемой волны) и на большом расстоянии $r \gg \lambda$ от центра ячейки в узком угловом секторе цилиндрическую волну можно рассматривать как плоскую волну с волновым числом $k$ в одномерном k-пространстве.

Для прямоугольной ячейки стоячие волны описывается суммой двух синусов:
\begin{equation}
\label{eq:waveStand}
h(r, t) = A_1 sin(kx)cos(\omega t)+A_2 sin(ky)cos(\omega t+ \phi)
\end{equation}
где \phi разность фаз между стоячими волнами в разных направлениях.

Бегущие волны волны будут задаваться выражением:
\begin{equation}
\label{eq:waveRun}
h(r, t) = A_1 sin(kx-\omega t)+A_2 sin(ky-\omega t+ \phi)
\end{equation}



%\section{Возбуждение волн в ячейке вертикальной тряской} %\label{subsect_boundary}

%Турбулентность в системе волн наряду с вихревой турбулентностью играет значительную роль во многих процессах, происходящих на Земле и во Вселенной. Она является объектом интенсивных исследований во многих системах: на поверхности океана, в атмосфере, в плазме [1]. 
%Турбулентноcть на поверхности воды в гравитационно-капиллярном интервале частот изучалась многими исследователями в течение нескольких последних десятилетий [2–5]. Для возбуждения волн использовали различные методики. Так, в [2] применяли специальные лопатки (мешалки), погруженные в жидкость. Однако в большинстве работ для генерации волн используют параметрическую неустойчивость поверхности жидкости, совершающей вынужденные вертикальные колебания с ускорениями выше некоторого порогового значения (неустойчивость Фарадея) [3–5]. Отличительной чертой этой методики является высокий уровень возбуждения волн сразу после возникновения неустойчивости на поверхности. Такая особенность методики возбуждения не позволяет работать с волнами малой амплитуды. Кроме того, как выяснилось, при сильном возбуждении наряду с нелинейным взаимодействием волн наблюдается генерация вихревого движения [6, 7]. Недавно в [8, 9] было показано, что завихренность формируется в результате взаимодействия нелинейных волн, имеющих непараллельные волновые векторы, т.е. в двумерном пространстве волновых векторов k. В [10] волны на поверхности цилиндрической ячейки возбуждали с помощью кольца, касающегося поверхности воды вблизи стенок ячейки. На поверхности возбуждалась только радиальная мода. В этом случае стоячие волны на поверхности описываются функцией Бесселя параметра $Rk$ ($R$ – радиус ячейки). Скаляр $k$ играет роль волнового числа: при больших значениях $R/\lambda$, ($\lambda = 2*\pi/k$ – длина возбуждаемой волны) и на большом расстоянии $r \gg \lambda$ от центра ячейки в узком угловом секторе цилиндрическую волну можно рассматривать как плоскую волну с волновым числом k в одномерном k-пространстве. Экспериментальные результаты [10] оказались в хорошем согласии с теорией слабой (волновой) турбулентности [1].

%В настоящем сообщении представлены экспериментальные результаты исследований турбулентности в системе волн на поверхности воды, возбуждаемых вертикальными колебаниями ячейки за счет краевого эффекта смачивания при ускорениях меньше порового значения возникновения неустойчивости Фарадея в цилиндрической ячейке, когда вихревое движение еще не наблюдается, и в квадратной ячейке, где вихревое движение при этих уровнях накачки хорошо развито.

%=======
\section{Способы возбуждения волн на поверхности жидкости}\label{p1_methodsExt}

%Турбулентность в системе волн наряду с вихревой турбулентностью играет значительную роль во многих процессах, происходящих на Земле и во Вселенной. Она является объектом интенсивных исследований во многих системах: на поверхности океана, в атмосфере, в плазме [1]. 
Турбулентноcть на поверхности воды в гравитационно-капиллярном интервале частот изучалась многими исследователями в течение нескольких последних десятилетий \cite{Falcon2007, Henry2000, Shats2010, Denissenko2007}. В лабораторных условиях волны на поверхности жидкости могут возбуждаться различными способами: при помощи волнопродукторов \cite{Havelock1929, Falcon2007}, электрическими силами, действующими на границу раздела жидкостей с разной диэлектрической проницаемостью \cite{Kalinichenko1982} или на поверхность заряженной жидкости \cite{Brazhnikov2002}, колебаниями сосуда с жидкостью в вертикальном направлении как целого \cite{Miles1990}. В последнем случае возможны два механизма возбуждения волн на поверхности воды. Первый осуществляется благодаря наличию мениска на границе ячейки. При вертикальных колебаниях равновесный радиус мениска меняется в зависимости от величины вертикального ускорения, что приводит к появлению на поверхности жидкости волн с частотой равной частоте вертикальных колебаний. 

Второй способ - в результате порогового развития параметрической неустойчивости на поверхности жидкости возникают волны, впервые описанные Фарадеем \cite{Faraday1831}. Так как неустойчивость параметрическая, то частота этих волн в два раза меньше частоты вертикальных колебаний. Во многих работах для генерации волн используют эту параметрическую неустойчивость Фарадея поверхности жидкости \cite{Henry2000, Shats2010, Denissenko2007}. Отличительной чертой этой методики является высокий уровень возбуждения волн сразу после возникновения неустойчивости на поверхности. Такая особенность методики возбуждения не позволяет работать с волнами малой амплитуды. При вертикальных колебаниях сосуда с жидкостью относительно низкой амплитуды параметрическая неустойчивость Фарадея не возникает, так как является пороговым эффектом. Поэтому для наблюдения спектров турбулентного каскада на поверхности воды использовалось возбуждение волн с помощью вертикальных колебаний с амплитудой ниже порога параметрической неустойчивости. 

Стоит так же отметить, что при возбуждении волн на поверхности воды с помощью мениска в цилиндрической ячейки возбуждаются только радиальные моды, а при развитии параметрической неустойчивости также происходит возбуждение азимутальной моды.



При наблюдении волновой турбулентности на поверхности жидкого водорода удобнее использовать возбуждение волн с помощью электрического поля. В этом случае амплитуда волн ограничивается напряжением пробоя и размерами оптического окна криостата, через которое с помощью лазерного луча ведется регистрация волн \cite{Brazhnikov2002}.



\section{Возбуждение вихревых течений неустойчивостью Фарадея}% \label{sect3_1}

Ранее было замечено, что при параметрическом возбуждении колебаний наряду с волновым движением на поверхности жидкости также наблюдается течение, демонстрирующее хаотическое поведение \cite{Ramshankar1990}. Впоследствии было показано, что это течение соленоидально и с увеличением амплитуды волн может оказаться достаточно интенсивным для формирования турбулентного каскада \cite{VonKameke2011, Francois2014} подобно обратному каскаду в двумерной турбулентности \cite{Kraichnan1967}. Несмотря на большое количество экспериментальных исследований, посвященных волнам Фарадея, природа возникновения в них течения до настоящего времени не выяснена. В работе \cite{Mesquita1992} это течение пытались описать как средний дрейф Стокса \cite{Stokes1847} для случайного волнового поля. Однако найденное в эксперименте значение коэффициента диффузии пассивного скаляра почти на порядок превышало теоретическое.

В данной работе будет предложен механизм нелинейной генерации вихревого движения волнами на поверхности жидкости, а так же представлены результаты экспериментов, которые демонстрируют, что формирование вихревого движения не является специфической чертой волн Фарадея, а связано с двухмерностью волнового движения на поверхности жидкости.

Для количественного изучения вихревых движений используется величина завихренности, определяемая как:

\begin{equation}
 \label{eq:defVort}
\Omega(x, y) = \frac{\partial V_x}{\partial y} - \frac{\partial V_y}{\partial x}
\end{equation}
где $V_x$, $V_y$ – компоненты скорости жидкости. 

%Кроме того, как выяснилось, при сильном возбуждении наряду с нелинейным взаимодействием волн наблюдается генерация вихревого движения \cite{Shats2005, VonKameke2011}.
% Недавно в [F5, F6] было показано, что завихренность формируется в результате взаимодействия нелинейных волн, имеющих непараллельные волновые векторы, т.е. в двумерном пространстве волновых векторов k. В \cite{Brazhnikov_liq_hydr} волны на поверхности цилиндрической ячейки возбуждали с помощью кольца, касающегося поверхности воды вблизи стенок ячейки. На поверхности возбуждалась только радиальная мода. В этом случае стоячие волны на поверхности описываются функцией Бесселя параметра $Rk$ ($R$ – радиус ячейки). Скаляр $k$ играет роль волнового числа: при больших значениях $R/\lambda$, ($\lambda = 2*\pi/k$ – длина возбуждаемой волны) и на большом расстоянии $r \gg \lambda$ от центра ячейки в узком угловом секторе цилиндрическую волну можно рассматривать как плоскую волну с волновым числом k в одномерном k-пространстве. Экспериментальные результаты \cite{Brazhnikov_liq_hydr} оказались в хорошем согласии с теорией слабой (волновой) турбулентности \cite{Zakharov}.

%В настоящем сообщении представлены экспериментальные результаты исследований турбулентности в системе волн на поверхности воды, возбуждаемых вертикальными колебаниями ячейки за счет краевого эффекта смачивания при ускорениях меньше порового значения возникновения неустойчивости Фарадея в цилиндрической ячейке, когда вихревое движение еще не наблюдается, и в квадратной ячейке, где вихревое движение при этих уровнях накачки хорошо развито.
%Гидродинамика жидкости со свободной поверхностью давно является предметом теоретических и экспериментальных исследований. 




\section{Законы сохранения} %\label{subsect1_lawSave}

При взаимодействии волн должны выполняться законы сохранения энергии и импульса. Для капиллярных волн трехволновые процессы могут удовлетворять законам сохранения импульса и энергии, поэтому закон дисперсии в капиллярной область называют распадным. Законы сохранения энергии и импульса для трехволнового процесса:
\begin{equation}
 \label{eq:saveOmega}
\omega_1 \pm \omega_2 = \omega_3
\end{equation}
\begin{equation}
 \label{eq:saveK}
\mbox{\boldmath$k_1$} \pm \mbox{\boldmath$k_2$} = \mbox{\boldmath$k_3$}
\end{equation}

Таким образом при возбуждении на поверхности жидкости волн в области низких частот может быть сформировано турбулентное состояние в котором поток энергии направлен от области низких частот(область накачки) в сторону больших частот. Теория слабой волновой турбулентности \cite{Zakharov} предсказывает, что основной вклад в перенос энергии по турбулентному капиллярному каскаду вносят как раз трехволновые процессы слияния волн. 

Стоит также отметить, что для гравитационных волн трехволновые процессы запрещенны из-за того, что не удовлетворяют законам сохранения энергии и импульса.

\section{Инерционный интервал турбулентного каскада}% \label{subsect1_1_3}

Вязкость для волн с разными частотами растет с увеличением частоты, таким образом энергия в турбулентном каскаде поверхностных волн передается в сторону больших частот до тех пор, пока вязкие потери не становятся сравнимы с потоком энергии.

В турбулентной системе можно выделить три характерные области: область накачки, в которой энергия приходит в систему, инерционный интервал, где энергия передается практически без потерь и область диссипации, где энергия покидает систему. 

В настоящий момент имеется довольно много теоретических и экспериментальных работ посвященных изучению инерционного интервала турбулентного каскада в различных системах<...>. Теория волновой турбулентности предсказывает степенное распределение энергии по шкале частот \cite{Zakharov}:
\begin{equation}
\label{eq:EOmega}
E_\omega \sim \omega^{-\alpha}
\end{equation}

С экспериментальной точки зрения удобно исследовать не распределение энергии по волновым векторам(или частотам) напрямую, парную корреляционную функцию отклонения поверхности от положения равновесия $I(\tau)=<(r, t+\tau)(r,t)>$, так как величину отклонения поверхности от положения равновесия можно непосредственно экспериментально измерить. Фурье образ парной корреляционную функцию отклонения поверхности от равновесного состояния связан с распределением энергии по частотам:
\begin{equation}
\label{eq:EOmegaI}
E_\omega \sim \omega^{-4/3}I_\omega
\end{equation}

Таким образом предсказывается степенная зависимость $I_\omega \sim \omega^{-m}$

В зависимости от характера накачки теория волновой турбулентности предсказывает различные показатели m. Для широкополосной накачки (когда ширина полосы накачки сопоставима или больше самой частоты накачки), предсказывается m = 17/6. При накачке узкополосным сигналом в спектре появляются равноудаленные пики, максимумы которых убывают c ростом частоты с показателем m = 23/6. Данные предсказания подтверждаются с помощью компьютерного моделирования \todo{[ссылки]}, так в экспериментальных исследованиях \cite{Brazhnikov_liq_hydr, Falcon2007}. Форма инерционного интервала в спектрах турбулентных каскадов в системе волн на поверхности воды, жидкого водорода и жидкого гелия экспериментально хорошо изучены в работах <...>. 

Стоит отметить, что одной из сложностью для экспериментального исследования и проведения вычислительных экспериментов по исследования турбулентных каскадов является степенное уменьшения энергии волны с ростом частоты. Так как величина показателя степени m находится в районе 2-3, а диапазон частот в котором существует турбулентный каскад может достигать нескольких декад, то для экспериментального исследования поведения турбулентной системы в достаточно широком диапазоне частот, необходимым для наблюдения развитого турбулентного каскада, требуется экспериментальное оборудование обладающее большим динамическим диапазоном. Таким образом экспериментальное исследование инерционного интервала и диссипативной области турбулентных каскадов было практически невозможно до появления широко распространенных АЦП с высоким динамическим диапазоном и достаточной высокой частотой оцифровки, а также развитием компьютерной техники для обработки полученных сигналов.

\section{Диссипативная область турбулентного каскада}% \label{subsect_disp}
Одна из важных величина характерезующих стационарный спектр турбулентного каскада является поток энергии $P$ от малах к большим частотам $\omega$. На больших частотах передаваемая энергия диссипирует из-за вязких потерь и турбулентный каскад разрушается. Следовательно для поддержания распределения ((\ref{eq:EOmega}) или (\ref{eq:EOmegaI}) постоянным во времени, система должна постоянно накачиваться энергией на низких частотах. Высокочастотная граница инерционного интервала $\omega_d$, где распределение (\ref{eq:EOmegaI}) еще остается верным, может быть оценена из предположения, что время вязкого затухания $\tau_\nu(\omega)$ и время нелинейного взаимодействия $\tau_{nl}(\omega)$ равны по порядку для волн на частоте $\omega_d$. Это предположение дает нам \cite{Malkin1984}:
\begin{equation}
% \label{eq:tauNu}
\omega_d	 \sim (\frac{P^{1/2}}{\nu})^{6/5} \sim (\eta_0^2\omega_0^{17/6}/\nu)^{6/5},
\end{equation}

%Несмотря на то, что в диссипативной области вязкое время превышает нелинейное и преобладают процессы затухания нелинейные процессы осуществляют существенное влияние на поведение спектра. Если волны в диссипативном интревале взаимодействуют в основном с ближайшими соседями, а не волнами из инерционного интервала, то распределение энергии по волнам в области диссипации становиться близким к экспоненциальному.

Где, $\nu_0^2$ - среднеквадратичная амплитуда волны на некоторой низкой частоте $\omega_0$, $\nu$ - кинематическая вязкость жидкости.

На частотах больших $\omega_d$ спектр зависит и от специфики затухания и от нелинейного взаимодействия. Когда волна в диссипативной области $\omega \gg \omega_d$ взаимодействует в основном с другими волнами диссипативной области, нежели волнами из инерционного интервала $\omega \ll \omega_d$ , волновое распределение будет похоже на экспоненциальное \cite{Ryzhenkova1990}. Детальное рассмотрение дает “квазипланковский” спектр для диссипативной области корреляционной функции \cite{Brazhnikov_IET}
\begin{equation}
% \label{eq:tauNu}
<\eta_\omega^2> \sim \omega^{-s} e^{-\omega/\omega_d},
\end{equation}			
где s - некая константа. Численные вычисления для дискретного кинетического уравнения \cite{Brazhnikov_IET} подтверждают экспоненциальную зависимость волнового числа заполнения в области сильного затухания.
	
%	Мы представляем первое экспериментальное наблюдение турбулентных спектров капиллярных волн в диссипативной области возбужденных низкочастотной случайной силой на поверхности жидкого водорода.


Слабая волновая теория предсказывает существование стационарного неравновесного состояния в системе взаимодействующих капиллярных волн - спектр Колмогорова-Захарова для спектральной плотности энергии $\varepsilon(k)$ волн в инерционном интервале.
\begin{equation}
% \label{eq:tauNu}
 \varepsilon(k) \sim P^{1/2}k^{-7/4}
\end{equation}

Используя соотношение между спектральной плотностью энергии $\varepsilon(k)$ и парной корреляционной функцией высоты поверхности $\eta(\mathbf{r}, t)$, и принимая во внимание закон дисперсии капиллярных волн $\omega(k) \sim k^{3/2}$, можно получить частотный спектр корреляционной функции $<\eta(t+\tau)\eta(t)>$:
\begin{equation}
% \label{eq:tauNu}
<\eta_\omega^2> \sim (\sigma k^2)^{-1} \varepsilon(k)(d\omega/dk)^{-1} \sim P^{1/2} \omega^{-17/6}.
\end{equation}



\section{Положение высокочастотной границы инерционного интервала}% \label{subsect_boundary}

Для определения частотной области в которой будет происходить основная диссипация энергии рассмотрим такие важные для изучении турбулентных каскадов характеристики как вязкое время и нелинейное время.

Вязкое время определяется как характерное время вязкого затухания волны с заданным волновым вектором.
\begin{equation}
% \label{eq:tauNu}
1/\tau_\nu = 2\nu k^2 = 2 \nu \omega^{4/3}(\sigma/\rho)^{2/3}
\end{equation}
Характерное время нелинейного взаимодействия капиллярных волн можно выразить через параметры жидкости и функцию распределения капиллярных волн $n(\omega)$:
\begin{equation}
% \label{eq:tauNu}
1/\tau_nl = |V_\omega|^2 n(\omega)
\end{equation}

Где $V(\omega) \approx (\sigma/\rho)^{3/2}\omega^{3/2}$ - коэффициент трехволнового нелинейности капиллярных волн

Функцию распределения капиллярных волн $n(\omega) \sim E_{\omega}\omega^{-1}$ 

Положение высокочастотной границы инерционного интервала можно оценить из предположения совпадения по порядку величины времени нелинейного взаимодействия волн и времени вязкого затухания на частоте границы.

Получаем оценку граничной частоты инерционного интервала $\omega_b \sim \eta_p^{2.4}$ для широкополосной накачки и $\omega_b \sim \eta_p^{4/3}$ для узкополосной накачки. Несмотря на то, что отдельные экспериментальные работы по измерению поведения положения границы инерционного интервала производились <>, работ направленных на создание общей картины поведения положения границы инерционного интервала для разных типов накачки и разной геометрии экспериментальной ячейки не было.


%Диссипативная область турбулентного каскада.
%Несмотря на то, что в диссипативной области вязкое время превышает нелинейное и преобладают процессы затухания нелинейные процессы осуществляют существенное влияние на поведение спектра. Если волны в диссипативном интревале взаимодействуют в основном с ближайшими соседями, а не волнами из инерционного интервала, то распределение энергии по волнам в области диссипации становиться близким к экспоненциальному. В \cite{Ryzhenkova1990} был проведен детальный анализ, который дал квазипланковский спектр корреляционной функций в диссипативной области:
%\begin{equation}
%% \label{eq:tauNu}
%P_\omega \sim e^{(-\omega/\omega_d)},
%\end{equation}
%где $\omega_d$ - характерная частота распределения.

%
%Напомним, что в турбулентном каскаде можно выделить три диапазона частот: область накачки, в которой в систему поступает энергия; инерционный интервал, где энергия передается в основном за счет нелинейного взаимодействия; и область диссипации, в которой энергия колебаний переходит в тепло. В инерционном интервале парная корреляционная функция отклонений поверхности от равновесия $I_\omega$ описывается степенной функцией:
%\begin{equation}
%% \label{eq:disper}
%I_\omega = \omega^{-m},
%\end{equation}
%где $\omega = 2 \pi f$ – частота, показатель степени m зависит от спектральной характеристики возбуждающей силы [1]: m = –17/6 для широкополосной накачки и m = –23/6 для узкополосной накачки. Теоретические оценки m были подтверждены в экспериментах с жидким водородом [11] и с водой [2].

%Край инерционного интервала определяется как характерная частота $f_b$, при которой нелинейное время взаимодействия волн сравнивается со временем вязкого затухания [10]. Положение частоты $f_b$ зависит как от свойств поверхности жидкости, так и от характеристик накачки: амплитуды, ширины полосы возбуждения. Теория предсказывает степенную зависимость положения края инерционного интервала $\omega_b$ от амплитуды накачки A:
%\begin{equation}
%% \label{eq:disper}
%f_b = A^{\beta}.
%\end{equation}

%Показатель степени $\beta$ зависит от типа накачки: при монохроматической накачке $\beta$ = 4/3, при широкополосной – $\beta$ = 12/5 [11]. Впервые край инерционного интервала в системе капиллярных волн наблюдался на поверхности жидкого водорода [10].
%Диссипативную область можно охарактеризовать двумя параметрами: положением края инерционного интервала $f_b$ и характерной частотой экспоненциального затухания $f_d$. На частотах $f > f_b$ степенной закон распределения $I_\omega$ переходит в экспоненциальное затухание [11]:
%\begin{equation}
%% \label{eq:disper}
%I_\omega = e^{(f/f_b)}.
%\end{equation}


Экспоненциальное падение в турбулентном каскаде на частотах выше $f_b$ наблюдали в системе капиллярно-гравитационных волн на поверхности жидкого водорода и гелия [12, 13]. В экспериментах с жидким водородом [12] в случае широкополосной накачки было показано, что характерная частота $f_d$ растет с увеличением амплитуды возбуждающей силы по степенному закону $f_d \sim A^{0.85}$.

Экспериментальные ячейки имеют конечные размеры, поэтому спектр поверхностных возбуждений носит дискретный характер. Это накладывает дополнительные ограничения на выполнение законов сохранения энергии и импульса [14]. В экспериментальных работах [15, 16] было показано, что выбором размеров ячейки при накачке на некоторых частотах можно организовать передачу энергии как на высокие, так и на низкие частоты.
Целью настоящей работы было проведение подробных исследований зависимостей высокочастотного края инерционного интервала $f_b$ и характерной частоты $f_d$ от амплитуды возбуждающей силы на поверхности воды в цилиндрической и квадратной ячейках при амплитудах накачки меньше порогового значения, при котором возникает параметрическая неустойчивость Фарадея.

%\section{Завихренность}% \label{sect4_1}
%Введение. Спектр волн на поверхности бесконечной глубокой жидкости описывается выражением:
%\begin{equation}
% \label{eq:disper}
%\omega^2 = (gk + \sigma/\rho k^3)
%\end{equation}
%где $\omega$ – круговая частота волны, $g$ – ускорение свободного падения, $k$ – волновой вектор, $\sigma$ – коэффициент поверхностного натяжения, $\rho$ – плотность жидкости. На поверхности воды волны длиной более $\lambda_0 = 2 \pi (\sigma \rho/g)^{1/2} = 1.7$ см принято считать гравитационными, а менее – капиллярными. Граничная частота равняется $\omega/2\pi = 13Гц$. На частотах 3 и 4Гц доминирующем в выражении (1) является первый гравитационный член. Он превосходит капиллярный член в 60 раз на частоте 3Гц и в 30 раз на частоте 4Гц, что оправдывает определение волн в этой области частот как гравитационные. Генерацию вихрей на поверхности воды, возбуждаемую фарадеевскими волнами, впервые наблюдали в работах [1, 2]. В нашей работе [3] мы наблюдали формирование вихревого течения поверхностными волнами в сосуде с жидкостью, совершающем гармонические колебания в вертикальном направлении при амплитудах накачки ниже порога параметрической неустойчивости. Было установлено, что генерация вихрей обусловлена взаимодействием нелинейных волн, распространяющихся под углом друг к другу. Теоретическая модель, описывающая формирование вихрей волнами, предложена в [4], в которой показано, что за генерацию завихренности на поверхности жидкости ответственно нелинейное взаимодействие поверхностных волн, распростроняющихся под углом друг к другу. Завихренность $\Omega$ на поверхности жидкости определяется как:
%
%\begin{equation}
% \label{eq:defVort}
%\Omega(x, y) = \frac{\partial V_x}{\partial y} - \frac{\partial V_y}{\partial x}
%\end{equation}
%где $V_x$, $V_y$ – компоненты скорости жидкости. Когда на поверхности жидкости возбуждаются стоячие волны на частоте $\omega$ в двух перпендикулярных направлениях, завихренность $\Omega$ зависит от амплитуды волн $H_1$, $H_2$, волнового вектора $k$ и разности фаз $\psi$ между волнами и описывается выражением:
%\begin{equation}
% %\label{eq:disperCap}
%\Omega = -(2+\sqrt{2}) sin(\psi) H_1 H_2 \omega k^2 sin(kx) sin(ky).
%\end{equation}
%Амплитуда завихренности $\Omega_0$ в случае стоячих волн определяется как:
%
%\begin{equation}
% %\label{eq:disperCap}
%\Omega = \Omega_0 sin(kx) sin(ky).
%\end{equation}
%(4) В случае волн, бегущих по поверхности, завих-
%ренность не зависит от разности фаз $\psi$ и описывается выражением:
%\begin{equation}
% %\label{eq:disperCap}
%\Omega = -(2+\sqrt{2}) sin(\psi) H_1 H_2 \omega k^2 sin(kx - ky).
%\end{equation}
%В работах [3, 4] были приведены экспериментальные результаты по генерации вихрей стоячими и бегущими капиллярными волнами на частотах около 40Гц в прямоугольной ячейке с размерами 5.0 $\times$ 4.9 см2. Длина волны на частоте 40Гц равняется приблизительно 0.7 см. Было установлено, что амплитудные зависимости завихренности для бегущих и стоячих капиллярных волн качественно хорошо описываются выражениями (3) и (5), но наблюдается расхождение в несколько раз по абсолютной величине завихренности $\Omega$. Согласно [4], зависимости (3) и (5) носят универсальный характер и должны выполняться как для капиллярных, так и для гравитационных волн. Поэтому целью настоящей работы было исследование амплитудной и фазовой зависимостей модуля завихренности при возбуждении поверхности гравитационными волнами, а также изучение распределения энергии по волновым векторам. В настоящем сообщении мы приводим экспериментальные результаты изучения генерации вихрей на поверхности воды гравитационными волнами на частотах 3 и 4Гц с длинами возбуждаемых резонансных мод, равными 17 и 9.7 см, соответственно.
%
%
%
%
%
%
%\clearpage