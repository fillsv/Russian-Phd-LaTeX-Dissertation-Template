%%% Основные сведения %%%
\newcommand{\thesisAuthor}             % Диссертация, ФИО автора
{%
    \texorpdfstring{% \texorpdfstring takes two arguments and uses the first for (La)TeX and the second for pdf
        Филатов Сергей Васильевич% так будет отображаться на титульном листе или в тексте, где будет использоваться переменная
    }{%
        Филатов, Сергей Васильевич% эта запись для свойств pdf-файла. В таком виде, если pdf будет обработан программами для сбора библиографических сведений, будет правильно представлена фамилия.
    }%
}
\newcommand{\thesisAuthorShort}        % Диссертация, ФИО автора инициалами
{С.В.~Филатов}

\newcommand{\thesisUdk}                % Диссертация, УДК
{532.59}
\newcommand{\thesisTitle}              % Диссертация, название
{\texorpdfstring{\MakeUppercase{Нелинейные волновые и вихревые движения на поверхности жидкости}}{Название диссертационной работы}}
\newcommand{\thesisSpecialtyNumber}    % Диссертация, специальность, номер
{\texorpdfstring{01.04.07}}
\newcommand{\thesisSpecialtyTitle}     % Диссертация, специальность, название
{физика конденсированного состояния}
\newcommand{\thesisDegree}             % Диссертация, ученая степень
{кандидата физико-математических наук}
\newcommand{\thesisDegreeShort}        % Диссертация, ученая степень, краткая запись
{канд. физ.-мат. наук}
\newcommand{\thesisCity}               % Диссертация, город написания диссертации
{Черноголовка}
\newcommand{\thesisYear}               % Диссертация, год написания диссертации
{2019}
\newcommand{\thesisOrganization}       % Диссертация, организация
{Федеральное государственное бюджетное учреждение науки Институт физики твердого тела Российской академии наук}
\newcommand{\thesisOrganizationShort}  % Диссертация, краткое название организации для доклада
{ИФТТ РАН}

\newcommand{\thesisInOrganization}     % Диссертация, организация в предложном падеже: Работа выполнена в ...
{Федеральном государственном бюджетном учреждении науки Институт физики твердого тела Российской академии наук}

\newcommand{\supervisorFio}            % Научный руководитель, ФИО
{Левченко Александр Алексеевич}
\newcommand{\supervisorRegalia}        % Научный руководитель, регалии
{доктор физико-математических наук, доцент}
\newcommand{\supervisorFioShort}       % Научный руководитель, ФИО
{{А.А.~Левченко}}
\newcommand{\supervisorRegaliaShort}   % Научный руководитель, регалии
{докт. физ.-мат. наук}


\newcommand{\opponentOneFio}           % Оппонент 1, ФИО
{Колоколов Игорь Валентинович}
\newcommand{\opponentOneRegalia}       % Оппонент 1, регалии
{доктор физико-математических наук, доцент}
\newcommand{\opponentOneJobPlace}      % Оппонент 1, место работы
{Федеральное государственное бюджетное учреждение науки Институт теоретической физики им. Л.Д. Ландау Российской академии наук}
\newcommand{\opponentOneJobPost}       % Оппонент 1, должность
{директор}

\newcommand{\opponentTwoFio}           % Оппонент 2, ФИО
{Болтнев Роман Евгеньевич}
\newcommand{\opponentTwoRegalia}       % Оппонент 2, регалии
{кандидат физико-математических наук}
\newcommand{\opponentTwoJobPlace}      % Оппонент 2, место работы
{Федеральное государственное бюджетное учреждение науки Объединённый институт высоких температур Российской академии наук}
\newcommand{\opponentTwoJobPost}       % Оппонент 2, должность
{старший научный сотрудник}

\newcommand{\leadingOrganizationTitle} % Ведущая организация, дополнительные строки
{Федеральное государственное бюджетное учреждение науки Институт теплофизики им. С.С. Кутателадзе Сибирского отделения Российской академии наук}

\newcommand{\defenseDate}              % Защита, дата
{\blank[\widthof{99999999999}] в 14:30 часов}
\newcommand{\defenseCouncilNumber}     % Защита, номер диссертационного совета
{Д.002.100.01}
\newcommand{\defenseCouncilTitle}      % Защита, учреждение диссертационного совета
{Федеральном  государственном бюджетном  учреждении  науки  Институт  физики  твёрдого  тела  Российской академии  наук}
\newcommand{\defenseCouncilAddress}    % Защита, адрес учреждение диссертационного совета
{142432,  Московская  область,  г.  Черноголовка,  ул. Академика Осипьяна, д. 2, ИФТТ РАН}
\newcommand{\defenseCouncilPhone}      % Телефон для справок
{\todo{+7~(000)~000-00-00}}

\newcommand{\defenseSecretaryFio}      % Секретарь диссертационного совета, ФИО
{Зверев В. Н.}
\newcommand{\defenseSecretaryRegalia}  % Секретарь диссертационного совета, регалии
{д-р~физ.-мат. наук}            % Для сокращений есть ГОСТы, например: ГОСТ Р 7.0.12-2011 + http://base.garant.ru/179724/#block_30000

\newcommand{\synopsisLibrary}          % Автореферат, название библиотеки
{ИФТТ РАН и на сайте диссертационного совета: http://www.issp.ac.ru/main/dis-council.html}
\newcommand{\synopsisDate}             % Автореферат, дата рассылки
{\blank[\widthof{999999999999}] 2019 года}

% To avoid conflict with beamer class use \providecommand
\providecommand{\keywords}%            % Ключевые слова для метаданных PDF диссертации и автореферата
{волновая турбулентность, нелинейные волны, вихревое движение}