
{\actuality}Обьект исследования и актуальность темы.

Турбулентной системой называется сильно возбужденная система со многими степенями свободы и направленным потоком энергии в пространстве степеней свободы. Турбулентность в системе волн наряду с вихревой турбулентностью играет значительную роль во многих процессах, происходящих как на Земле, так и во Вселенной. Она является объектом  интенсивных исследований во многих системах: на поверхности океана, в атмосфере, в плазме, и т.д.. Турбулентые вихревые процессы так же играют значительную роль в определении погодных и климатических явлений. Одним из ключевых вопрос в понимании турбулентных явлений является вопрос передачи и диссипации энергии и импульса. 
Несмотря на то, что турбулентные и вихревые системы изучаются многими исследователями в течение нескольких последних десятилетий, сложность исследуемых объектов и многообразие возникающих эффектов оставляют открытыми многие вопросы, в частности, вопросы касающиеся взаимодействия систем, передачи и диссипации энергии. 
В данной работе представлено исследование диссипации энергии в слаботурбулентной системе капиллярных волн на поверхностях воды и жидкого водорода и исследования вихревого движения, возникающего как результат слабонелинейного взаимодействия волн на поверхности воды.
% {\progress} 
% Этот раздел должен быть отдельным структурным элементом по
% ГОСТ, но он, как правило, включается в описание актуальности
% темы. Нужен он отдельным структурынм элемементом или нет ---
% смотрите другие диссертации вашего совета, скорее всего не нужен.

{\aim} данной работы является: 
\begin{enumerate}
	\item Исследование диссипативной области стационарных турбулентных спектров на поверхности жидкости.
	\item Исследование положения края инерционного интервала турбулентного каскада на поверхности жидкости
	\item Исследование генерации вихревого движения волнами на поверхности жидкости.
\end{enumerate}

Для~достижения поставленной цели необходимо было решить следующие {\tasks}:
\begin{enumerate}
	\item Исследование диссипативной области турбулентного каскада в систем капиллярных волн на поверхностях жидкого водорода и воды
	\item Исследование положения края инерционного интервала турбулентного каскада в систем капиллярных волн на поверхностях воды
%	\item Разработка метода регистрации стоячих волн на поверхности прозрачной жидкости.???
	\item Исследование формирования вихревого движения в волновой системе на поверхности воды.
\end{enumerate}

{\novelty}
\begin{enumerate}
	\item Впервые проведено экспериментальное наблюдение «квази-планкововского» спектра в системе капиллярных волн на поверхности жидкого водорода.
%	\item Разработан новый метод регистрации стоячих волн на поверхности прозрачной жидкости?
	\item Предложен и экспериментально подтвержден новый механизм формирования вихревого движения волнами на поверхности жидкости.
	\item Экспериментально исследован механизм формирования вихревого движения волнами на поверхности жидкости.
	
\end{enumerate}

{\influence} \ldots

{\methods} \ldots

{\defpositions}
\begin{enumerate}
	\item Экспериментальное наблюдение «квази-планкововского» спектра в системе капиллярных волн на поверхности жидкого водорода.
	\item Исследование поведения положния края ИИ в зависимости от амплитуды накачки в случаях широкополоской и узкополосной накачки.
	\item Исследование «поведения диссипативной области» от амплитуды накачки в случаях широкополоской и узкополосной накачки.
%	\item Разработка метода регистрации стоячих волн на поверхности прозрачной жидкости.
	\item Исследование формирования вихревого движения волнами на поверхности жидкости.
	\item Предложение и экспериментальное подтверждение механизма формирования вихревого движения волнами на пов-ти жидкости
%	\item Исследование динамик формирования и затухания вихревого движения.
\end{enumerate}

%В папке Documents можно ознакомиться в решением совета из Томского ГУ
%в файле \verb+Def_positions.pdf+, где обоснованно даются рекомендации
%по формулировкам защищаемых положений. 

{\reliability} полученных результатов обеспечивается \ldots \ Результаты находятся в соответствии с результатами, полученными другими авторами.


{\probation}
Основные результаты работы докладывались~на:
перечисление основных конференций, симпозиумов и~т.\:п.

{\contribution} Все экспериментальные данные представленные в диссертационной работе были получены при непосредственном участии автора данной работы.

%\publications\ Основные результаты по теме диссертации изложены в ХХ печатных изданиях~\cite{Sokolov,Gaidaenko,Lermontov,Management},
%Х из которых изданы в журналах, рекомендованных ВАК~\cite{Sokolov,Gaidaenko}, 
%ХХ --- в тезисах докладов~\cite{Lermontov,Management}.

%\ifnumequal{\value{bibliosel}}{0}{% Встроенная реализация с загрузкой файла через движок bibtex8
%    \publications\
     Основные результаты по теме диссертации изложены в XX печатных изданиях, 
    X из которых изданы в журналах, рекомендованных ВАК, 
    X "--- в тезисах докладов.%
%}{% Реализация пакетом biblatex через движок biber
%Сделана отдельная секция, чтобы не отображались в списке цитированных материалов
%    \begin{refsection}[vak,papers,conf]% Подсчет и нумерация авторских работ. Засчитываются только те, которые были прописаны внутри \nocite{}.
%        \printbibliography[heading=countauthorvak, env=countauthorvak, keyword=biblioauthorvak, section=1]%
%        \printbibliography[heading=countauthorconf, env=countauthorconf, keyword=biblioauthorconf, section=1]%
%        \printbibliography[heading=countauthornotvak, env=countauthornotvak, keyword=biblioauthornotvak, section=1]%
%        \printbibliography[heading=countauthor, env=countauthor, keyword=biblioauthor, section=1]%
%        \nocite{%Порядок перечисления в этом блоке определяет порядок вывода в несгруппированном списке публикаций автора
%                vakbib1,vakbib2,%
%                confbib1,confbib2,%
%                bib1,bib2,%
%        }
%        \publications\ Основные результаты по теме диссертации изложены в \arabic{citeauthor} печатных изданиях, 
%        \arabic{citeauthorvak} из которых изданы в журналах, рекомендованных ВАК, 
%        \arabic{citeauthorconf} "--- в тезисах докладов.
%    \end{refsection}
%    \begin{refsection}[vak,papers,conf]%Блок, позволяющий отобрать из всех работ автора наиболее значимые, и только их вывести в автореферате, но считать в блоке выше общее число работ
%        \printbibliography[heading=countauthorvak, env=countauthorvak, keyword=biblioauthorvak, section=2]%
%        \printbibliography[heading=countauthornotvak, env=countauthornotvak, keyword=biblioauthornotvak, section=2]%
%        \printbibliography[heading=countauthorconf, env=countauthorconf, keyword=biblioauthorconf, section=2]%
%        \printbibliography[heading=countauthor, env=countauthor, keyword=biblioauthor, section=2]%
%        \nocite{vakbib2}%vak
%        \nocite{bib1}%notvak
%        \nocite{confbib1}%conf
%    \end{refsection}
%}
%При использовании пакета \verb!biblatex! для автоматического подсчёта
%количества публикаций автора по теме диссертации, необходимо
%их здесь перечислить с использованием команды \verb!\nocite!.
    

