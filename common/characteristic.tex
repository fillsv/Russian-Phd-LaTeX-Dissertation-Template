
{\actuality}Обьект исследования и актуальность темы.

Турбулентной системой называется сильно возбужденная система со многими степенями свободы и направленным потоком энергии в пространстве степеней свободы. Волновая турбулентность наряду с вихревой турбулентностью играет значительную роль во многих процессах, происходящих как на Земле, так и во Вселенной. Она является объектом  интенсивных исследований во многих явлениях: волновых и вихревых процессах на поверхности океана, в атмосфере, в плазме, и т.д.. Турбулентные вихревые процессы играют значительную роль в определении погодных и климатических явлений. Одним из ключевых вопрос в понимании турбулентных явлений является вопрос передачи и диссипации энергии. 

Несмотря на то, что турбулентные и вихревые системы изучаются многими исследователями в течение нескольких последних десятилетий, сложность исследуемых объектов и многообразие возникающих эффектов оставляют открытыми многие вопросы, в частности, вопросы касающиеся взаимодействия систем, передачи и диссипации энергии. 
В данной работе представлено исследование диссипации энергии в слаботурбулентной системе капиллярных волн на поверхностях воды и жидкого водорода и исследования вихревого движения, возникающего как результат слабонелинейного взаимодействия волн на поверхности воды.
% {\progress} 
% Этот раздел должен быть отдельным структурным элементом по
% ГОСТ, но он, как правило, включается в описание актуальности
% темы. Нужен он отдельным структурынм элемементом или нет ---
% смотрите другие диссертации вашего совета, скорее всего не нужен.

{\aim} данной работы является: 
\begin{enumerate}
	\item Исследование диссипативной области стационарных турбулентных спектров в системе волн на поверхности жидкости.
	\item Исследование положения края инерционного интервала турбулентного каскада в системе волн на поверхности жидкости
	\item Исследование генерации вихревого движения волнами на поверхности жидкости.
\end{enumerate}

Для~достижения поставленной цели необходимо было решить следующие {\tasks}:
\begin{enumerate}
	\item Измерение распределения энергии в высокочастотной области турбулентного каскада в системе капиллярных волн на поверхности жидкого водорода при разных интенсивностях возбуждения волн.
	\item Измерение распределения энергии в высокочастотной области турбулентного каскада в системе капиллярных волн на поверхности воды при разных интенсивностях возбуждения волн
	\item Измерение распределения энергии в высокочастотной области турбулентного каскада в системе капиллярных волн на поверхности воды в экспрементальных ячейках различной геометрии.

%	Исследование диссипативной области турбулентного каскада в системе капиллярных волн на поверхностях жидкого водорода
%	\item Исследование положения края инерционного интервала турбулентного каскада в систем капиллярных волн на поверхностях воды
	\item Создание экспериментальных установок для исследования генерации вихревого движение в системе капиллярных волн и в системе гравитационных волн.
	\item Исследование условий формирования вихревого движения волновой системой на поверхности воды.
\end{enumerate}

{\novelty}
\begin{enumerate}
	\item Впервые проведено экспериментальное наблюдение «квази-планкововского» спектра в системе капиллярных волн на поверхности жидкого водорода.
%	\item Разработан новый метод регистрации стоячих волн на поверхности прозрачной жидкости?
	\item Предложен и экспериментально подтвержден новый механизм формирования вихревого движения волнами на поверхности жидкости.
	\item Экспериментально исследован механизм формирования вихревого движения волнами на поверхности жидкости.
	
\end{enumerate}

{\influence} \ldots

{\methods} \ldots


{\defpositions}
\begin{enumerate}
	\item Впервые экспериментально наблюден «квази-планкововский» спектр в турбулентом каскаде системы капиллярных волн на поверхности жидкого водорода.
	\item Экспериментально выявлена степенная зависимость характерной частоты экспоненциального спада энергии в диссипативной области турбулентного каскада системы капиллярных волн на поверхности жидкого водорода от амплитуды широкополосной накачки.
%	\item Экспериментально выявлены степенные зависимости характерной частоты экспоненциального спада энергии в диссипативной области турбулентного каскада системы капиллярных волн на поверхности воды от амплитуды как широкополосной, так и монохроматической накачки.
	\item Экспериментально показано, что при возбуждении турбулентного состояния на поверхности воды монохроматической или широкополосной накачкой характерная частота высокочастотного края инерционного интервала и характерная частота экспоненциального спада энергии в диссипативной области повышаются с ростом амплитуды накачки по степенному закону.% Экспериментально найдены показатели степени для случаев монохроматической и широкополосной накачки.

%	\item Разработка метода регистрации стоячих волн на поверхности прозрачной жидкости.
	\item Экспериментально показано, что возникновение вихревого движения происходит из-за взамодействия волн, распространяющихся под углом друг к другу.
	\item Экспериментально исследован механизм формирования вихревого движения двумя перпендикулярными стоячими волнами как в случае капиллярных, так и в случае гравитационных волн на поверхности воды.
	\item Экспериментально измерена амплитудная зависимость вихревого движения от амплитуды и относительной фазы двух перпендикулярных стоячих волн.
	\item Экспериментально наблюдена передача энергии от системы вихрей, образующих решетку, в большие вихревые масштабы?
%	\item Исследование динамик формирования и затухания вихревого движения.
\end{enumerate}

%В папке Documents можно ознакомиться в решением совета из Томского ГУ
%в файле \verb+Def_positions.pdf+, где обоснованно даются рекомендации
%по формулировкам защищаемых положений. 

{\reliability} полученных результатов обеспечивается \ldots \ Результаты находятся в соответствии с результатами, полученными другими авторами.


{\probation}
Основные результаты работы докладывались~на:
\begin{enumerate}
	\item XXIV научная сессия Совета РАН по нелинейной динамике (Москва, Россия, 2015)
	\item Научная школа "Нелинейные волны 2016" (Нижний Новгород, Россия, 2016)
	\item VIII-th International Conference "SOLITONS, COLLAPSES AND TURBULENCE: Achievements, Developments and Perspectives" (SCT-17) in honor of Evgeny Kuznetsov's 70th birthday (Черноголовка, Россия, 2017)
%	\item СС в г. Турку?
\end{enumerate}


{\contribution} Все экспериментальные данные представленные в диссертационной работе были получены при непосредственном участии автора данной работы.

%\publications\ Основные результаты по теме диссертации изложены в ХХ печатных изданиях~\cite{Sokolov,Gaidaenko,Lermontov,Management},
%Х из которых изданы в журналах, рекомендованных ВАК~\cite{Sokolov,Gaidaenko}, 
%ХХ --- в тезисах докладов~\cite{Lermontov,Management}.

%\ifnumequal{\value{bibliosel}}{0}{% Встроенная реализация с загрузкой файла через движок bibtex8
%    \publications\
     Основные результаты по теме диссертации изложены в XX печатных изданиях, 
    X из которых изданы в журналах, рекомендованных ВАК, 
    X "--- в тезисах докладов.%
%}{% Реализация пакетом biblatex через движок biber
%Сделана отдельная секция, чтобы не отображались в списке цитированных материалов
%    \begin{refsection}[vak,papers,conf]% Подсчет и нумерация авторских работ. Засчитываются только те, которые были прописаны внутри \nocite{}.
%        \printbibliography[heading=countauthorvak, env=countauthorvak, keyword=biblioauthorvak, section=1]%
%        \printbibliography[heading=countauthorconf, env=countauthorconf, keyword=biblioauthorconf, section=1]%
%        \printbibliography[heading=countauthornotvak, env=countauthornotvak, keyword=biblioauthornotvak, section=1]%
%        \printbibliography[heading=countauthor, env=countauthor, keyword=biblioauthor, section=1]%
%        \nocite{%Порядок перечисления в этом блоке определяет порядок вывода в несгруппированном списке публикаций автора
%                vakbib1,vakbib2,%
%                confbib1,confbib2,%
%                bib1,bib2,%
%        }
%        \publications\ Основные результаты по теме диссертации изложены в \arabic{citeauthor} печатных изданиях, 
%        \arabic{citeauthorvak} из которых изданы в журналах, рекомендованных ВАК, 
%        \arabic{citeauthorconf} "--- в тезисах докладов.
%    \end{refsection}
%    \begin{refsection}[vak,papers,conf]%Блок, позволяющий отобрать из всех работ автора наиболее значимые, и только их вывести в автореферате, но считать в блоке выше общее число работ
%        \printbibliography[heading=countauthorvak, env=countauthorvak, keyword=biblioauthorvak, section=2]%
%        \printbibliography[heading=countauthornotvak, env=countauthornotvak, keyword=biblioauthornotvak, section=2]%
%        \printbibliography[heading=countauthorconf, env=countauthorconf, keyword=biblioauthorconf, section=2]%
%        \printbibliography[heading=countauthor, env=countauthor, keyword=biblioauthor, section=2]%
%        \nocite{vakbib2}%vak
%        \nocite{bib1}%notvak
%        \nocite{confbib1}%conf
%    \end{refsection}
%}
%При использовании пакета \verb!biblatex! для автоматического подсчёта
%количества публикаций автора по теме диссертации, необходимо
%их здесь перечислить с использованием команды \verb!\nocite!.
    

